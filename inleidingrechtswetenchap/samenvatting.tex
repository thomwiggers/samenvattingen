\documentclass{article}

\usepackage[utf8]{inputenc}
\usepackage{microtype}
\usepackage{hyperref}

\author{Thom Wiggers}
\title{{\Huge Samenvatting}\\\ \\ Inleiding in het Nederlandse Recht, 17e druk, \\ \large J.~W.~P.~Verheugt}

\begin{document}
\maketitle

\section{Recht in het Algemeen}
\label{h1}

\subsection{Rechtsbronnen}

\emph{Positief recht} is het geheel van geldende rechtsregels. Dit staat ook
wel bekend als \emph{Objectief recht}. Dit zijn dus algemeen geldende
bepalingen. Een \emph{subjectief recht} is een bevoegdheid die iemand in een
concreet geval aan een objectief recht ontleent.

\emph{Rechtsbronnen} zijn bronnen waaruit het recht voorkomt. In Nederland
worden de volgende rechtsbronnen onderscheiden:

\begin{itemize}
  \item de wet
  \item jurisprudentie
  \item de gewoonte
  \item verdragen en besluiten van volkenrechtelijke organisaties.
\end{itemize}

Er zijn verschillende onderscheidingen mogelijk tussen de verschillende
onderdelen van het positieve recht. Daarover gaat de rest van dit hoofdstuk.

\subsection{Nationaal en internationaal recht}

\emph{Soevereiniteit} betekent dat elk land in zijn grondgebied de omvang en
inhoud van zijn nationale rechtsstelsel mag bepalen. Hierbij hoeft hij geen
invloeden van buitenaf te dulden.

De relaties tussen landen worden beheersd door een door de eeuwen heen ontstaan
stelsel van normen. Die normen bestaan deels door gewoonte, deels door
verdragen. Dit geheel wordt het \emph{Volkenrecht} genoemd.

\emph{Verdragen} zijn schriftelijke, bindende regelingen tussen staten
onderling of tussen staten en volkenrechtelijke organisaties. Er zijn
verschillende typen verdragen. Zo is een veelvoorkomende soort verdragen
waarbij enkel wederzijdse afspraken gemaakt wordt tussen regeringen,
bijvoorbeeld over politiesamenwerking in grensregio's. Een ander type verdrag
kan de deelnemende staten verplichten bepaalde soorten wetgeving toe te passen
of te maken. Zie hiervoor ook hoofdstuk \ref{h15}. Er zijn ook verdragen die
ingrijpen op de soevereiniteit van landen door directe werking te hebben zonder
tussenkomst van de wetgever. Dit is bijvoorbeeld het EVRM. Nederland heeft een
\emph{monistisch} systeem, dat wil zegen dat rechtsregels uit een verdrag
zonder omzetting in een nationale wet deel kunnen uitmaken van het nationale
recht. Dit is vastgelegd in artikel 93 Gw. Een laatste type verdrag bestaat uit
regelingen waarbij besturende, wetgevende en rechtssprekende bevoegdheid wordt
toegewezen aan supranationale organisaties. Een voorbeeld van zo'n verdrag is
het verdrag betreffende de werking van de Europese Unie. De regering kan
dergelijke inperkingen van de soevereiniteit van Nederland doen overeenkomstig
artikel 92 Gw.

Een regel van een verdrag of een besluit van internationale herkomst gaat boven
een regel van nationaal recht. Zie artikel 94 Gw.

\subsection{Materieel en formeel recht}

\emph{Materieel recht} zijn de regels over de rechten en plichten van personen
in hun onderlinge verkeer. \emph{Formeel recht} zijn de regels over procederen
voor de rechter. Bijvoorbeeld Sr vs Sv.

\subsection{Rechtsgebieden}

\subsubsection{Staatsrecht}

Het staatrecht bevat de regels die betrekking hebben op de organisaties van de
staat en zijn organen en de bevoegdheden van die organen. Ook omvat het de
verhoudingen van de burgers tot de staat en hoe burgers invloed kunnen
uitoefenen op het functioneren van de overheid.

De \emph{Grondwet} is het belangrijkste fundament in het staatsrecht. Deze
bevat eerst een aantal \emph{Grondrechten} zoals het gelijkheidsbeginsel.
Vervolgens beschrijft de grondwet de inrichting van de belangrijkste
overheidsorganen. Ook worden een aantal onderwerpen aangegeven die in andere
wetten moeten worden uitgewerkt. 

\subsubsection{Bestuursrecht}

Bestuursrecht gaat over de rechtsverhouding tussen een burger en de overheid.
Een \emph{beschikking} is een besluit van een bestuursorgaan dat gevolgen
vaststelt voor een specifieke (rechts-)persoon.

\subsubsection{Strafrecht}

Het strafrecht stelt bepaalde zaken strafbaar en draagt de handhaving daarvan
op aan de overheid. In artikel 16 Gw en artikel 1 lid 1 Sr is vastgelegd dat
niets strafbaar kan worden gesteld anders dan door vooraf wettelijke
strafbaarstelling.

\subsubsection{Burgerlijk recht}

Het burgerlijk recht omvat het personen- en familierecht, het
rechtsbersonenrecht en het vermogensrecht. 

\subsection{Publiekrecht en privaatrecht}

Het publiekrecht omvat het staatsrecht, het bestuursrecht (formeel:
bestuursprocesrecht), het strafrecht (formeel: strafprocesrecht), en het
volkenrecht.

Het privaatrecht omvat het personen- en familierecht, het vermogensrecht en het
rechtspersonenrecht. Het formele privaatrecht is het burgerlijk procesrecht.

\subsection{Andere indelingen}

Er kan onderscheid gemaakt worden tussen \emph{geschreven en ongeschreven recht}:
geschreven recht zijn wetten en verdragen, ongeschreven recht zijn
jurisprudentie en de gewoonte. 

Ook wordt het recht wel eens ingedeeld naar functie: bijvoorbeeld
Wegenverkeerswet.

\section{Recht en staat}
\label{h2}

\subsection{Trias Politica}

Montesquieu heeft de \emph{scheiding van machten bedacht}, waarbij de
rechtssprekende, wetgevende en besturende machten (organen) strikt gescheiden
zijn. Aanvullend moeten er \emph{``checks and balances''} zijn: respectievelijk
onderlinge toezichtstaken en gedeelde bevoegdheden. 

Een invloed van Montesquieu is het \emph{legisme}, de filosofie dat de wet,
vastgesteld door de wetgever, de enige rechtsbron is. Een ideaal van het
legisme is de \emph{codificatiegedachte}, de gedachte om al het recht in
systematische wetboeken vast te leggen.

\subsection{De Trias Politica in Nederland}

In Nederland is geen strikte implementatie van de trias politica. De
doelstelling van een machtsevenwicht lijkt echter wel bereikt.

\subsubsection{De wetgevende macht}
 
In Nederland is de wetgevende macht opgedragen aan zowel de Staten-Generaal
als aan de regering, zie artikel 81 Gw. 

\subsubsection{De uitvoerende macht}

De uitvoerende macht is opgedragen aan de regering, die bestaat uit de Koning
en de ministers. (artikel 42 Gw ev) De Staten-Generaal oefent toezicht uit op
de regering. 

Besluiten van de regering heten \emph{Koninklijk Besluit (KB)}. Deze kunnen twee 
vormen hebben: een beschikking of wetgeving. Als een KB wetgeving betreft wordt deze 
ook wel een \emph{Algemene Maatregel van Bestuur (AMvB)} genoemd. De regering 
heeft een speciale wetgevende bevoegdheid volgens artikel 81 Gw.

\subsubsection{De rechtsprekende macht}

De rechtsprekende macht is opgedragen aan de rechterlijke macht en de gerechten 
die niet tot de rechterlijke macht horen, zie artikel 112 e.v. Gw. Het meeste 
over de rechterlijke macht is geregeld in de Wet RO. 

In een bepaald aantal gevallen (bvb wijzigen voornaam) doet de rechter ook 
beschikkingen, maar de rechter mag nooit wetgeving maken: dit wordt door de
Artikel 12 Wet AB uitdrukkelijk verboden.

Ook mag de rechter de wetgever niet bevelen om wetgeving tot stand te brengen:
zie \textbf{HR Nitraatrichtlijn}.

De onafhankelijkheid van de rechter wordt onder meer gegarandeerd doordat de
rechter niet ondergeschikt aan enig overheidsorgaan is. Ook zijn rechters geen
verantwoording verschuldigd aan collega's. 

\subsubsection{Decentralisatie}

Het staatrecht kent ook een tweede scheiding van macht, die tussen de centrale
overheid en lagere overheden. Dit zijn voornamelijk \emph{Provincies,
gemeenten, waterschappen}

Er zijn twee vormen van decentralisatie: \emph{terretoriale decentralisatie} en
\emph{functionele decentralisatie}. Voorbeeld van deze laatste vorm zijn
productschappen. 

Decentralisatie wordt ingevuld op twee manieren: \emph{autonomie} bestaat als
de lagere overheid zelfstandig zaken mag regelen, van \emph{medebewind} is
sprake als het orgaan mee moet werken aan de verwezelijking van wat op centraal
niveau is besloten. 

\subsection{Nederland: een parlementaire democratie}

De \emph{ministeri\"ele verantwoordelijkheid}, betekent dat ministers
verantwoording af moeten leggen aan de kamers. De (ongeschreven)
\emph{vertrouwensregel} komt neer op dat een Kabinet haar ontslag aanbiedt
zodra het vertrouwen in hen onherroepelijk is opgezegd. In theorie mag het
kabinet de kamer ontbinden, maar dat wordt meestal slechts gedaan als
verkiezingen toch al in zicht zijn, en nooit zelfstandig.

Het \emph{Recht van interpellatie} betekent dat beide kamers schriftelijk of
mondeling vragen mogen stellen aan de ministers of staatssecretarissen. De
minister kan dit slechts wijzigen als de verlangde inlichtingen in strijd zijn
met het landsbelang. Een minister mag dit slechts wijzigen ``bij hoge
uitzondering en alleen met een toerijkende motivering'' (\textbf{HR Mink K.~
II}). Een eenvoudiger vorm van dit recht is het \emph{Vragenrecht}: deze worden
typisch behandeld tijdens het wekelijkse vragenuurtje.

Als er naar oordeel van Tweede Kamerleden bepaalde maatregelen nodig zijn,
kunnen deze een motie aan de Kamer voorleggen. 

Het laatste belangrijke controlerecht is het \emph{Recht van enqu\^ete} waarbij
de kamers onderzoek mogen doen.

\subsection{Grondrechten}

De grondrechten komen in Nederland met name uit het \emph{EVRM}, het
\emph{Internationaal Verdrag inzake de burgerrechten en politieke rechten
(IVBPR of BUPO)} en het \emph{Europees Sociaal Handvest}. Deze verdragen
hebben merendeels (behalve ESH) een eenieder verbindende werking. 

De klassieke grondrechten zijn vooral de vrijheidsrechten, politieke rechten
en gelijkheidsrechten. 

De grondrechten werken op de relatie tussen de overheid en de burgers en 
hebben zo een \emph{verticale werking}. De rechten die relaties tussen
burgers onderling bepalen hebben een \emph{horizontale werking}.

De sociale grondrechten dienen vooral als een opdracht aan de overheid maar 
kunnen niet in rechte worden gehaald.

Zie ook \textbf{HR Nederland Ontwapent}: de rechter biedt burgers verregaande
bescherming wat betreft grondrechten.

\textbf{HR Rechterlijk Uitzendverbod}: de rechter mag wel uitingen die jegens
anderen onrechtmatig zijn verbieden.

\section{De Wetgeving}

\subsection{De wet als rechtsbron}

Een wet bevat vrijwel altijd \emph{algemeen verbindende voorschriften}. Dit
zijn voorschriften die in een onbepaald aantal gevallen op een onbepaald aantal
mensen van toepassing zijn. 

Er zijn verschillende wetgevers: Regering en Staten-Generaal maken samen
wetten. De regering mag ook AMvBs uitvaardigen. Ministers kunnen ministeri\"ele
regelingen uitvaardigen. Lagere overheden kunnen verordeningen uitvaardigen. 

\subsubsection{Wetten in formele en materi\"ele zin}

Een wet in formele zin is een besluit van regering en Staten-Generaal. Een wet
in materi\"ele zin is een algemeen verbindend voorschrift en kan dus ook van
lagere wetgevers afkomstig zijn.

Een aantal zaken mag alleen bij formele wet besloten worden. 

Wetten in formele zin hebben altijd ``wet'' in de naam. AMvBs heten vaak
``regelement'' of ``besluit''. Ministeri\"ele regelingen heten ``Regeling'' en
algemeen verbindende voorschriften van decentrale wetgevers heten
``verordening''. 

\subsubsection{Wetgeving van de centrale overheid}

\emph{Organieke wetten} zijn wetten die worden voorgeschreven door de grondwet. 

Het proces van formele wetgeving is deze:

\begin{enumerate}
  \item Wetsvoorstel
  \item Raad van State brengt openbaar advies uit. Soms leidt dit tot een Nota van Wijziging.
  \item Indiening wet bij Tweede Kamer. Hierbij is een Memorie van Toelichting toegevoegd.
  \item De Tweede Kamer bespreekt het wetsvoorstel in commissies. Dit levert vragen op aan de minister
    die vervolgens een Memorie van Antwoord produceert.
  \item Op de agenda Tweede Kamer gezet. Behandeling in openbare vergadering. Tweede Kamerleden hebben 
    een recht van amendement. Over amendamenten wordt afzonderlijk gestemd. 
  \item De eerste kamer kan het voorstel vervolgens aanvaarden of verwerpen.
  \item Koning bekrachtigt (ondertekent) het. 
\end{enumerate}


Algemene maatregelen van bestuur worden bij \emph{Koninklijk besluit}
vastgesteld (artikel 88 Gw). Vaak zijn AMvBs bedoeld als invulling van
wetsmaatregelen, die deze bevoegdheid aan de overheid delegeren.

Ministeriële regelingen kunnen betrekkelijk makkelijk worden vastgesteld, maar
kunnnen slechts na delegatie worden vastgesteld. 


\subsection{De bevoegdheid tot wetgeving}

Een orgaan dat een voorschrift vaststelt, moet wel tot wetgeving bevoegd
zijn. Dit kan door attributie en delegatie. \emph{Attributie} is het 
creëren van een bevoegdheid om een bepaald soort regels vast te stellen
en die bevoegdheid toe te wijzen aan een bepaald orgaan. \emph{Delegatie}
is het overdragen van een al bestaande bevoegdheid om nadere regels vast 
te stellen. 

\subsubsection{Attributie}

De volgende wetgevende bevoegdheden worden door de Grondwet toegekend:

\begin{description}
  \item[Regering en Staten-Generaal] Artikel 81 Grondwet
  \item[Regering] AMvBs: Artikel 89 lid 1 Grondwet
  \item[Provinciale Staten en de Gemeenteraad] Verordeningen: Artikel 127 Grondwet.
  \item[Waterschapsbesturen] Verordeningen: Artikel 133 lid 2 Grondwet.
  \item[Besturen van openbare lichamen voor beroep en bedrijf]Verordeningen:
    artikel 134 Gw.
\end{description}

Dit wordt vervolgens door de gemeentewet, de waterschapswet en de provinciewet
verder uitgewerkt. Alleen aan de bevoegdheden van de formele wetgever zijn geen
grenzen gesteld, zie artikel 81 Gw. Als een gedecentraliseerde overheid een
verordening vaststelt waarin een hogere overheid later voorziet, vervalt deze
bevoegdheid van rechtswege. Zie artikel 122 Gemeentewet en Artikel 119
provinciewet. 

\subsubsection{Delegatie}

Van delegatie is sprake als een orgaan opdracht krijgt van een hoger tot 
wetgeving bevoegd orgaan om wetgeving vast te stellen binnen het kader 
van een wettelijke regeling. 

Subdelegatie is het weer doorgeven van deze bevoegdheid. Dit is niet altijd mogelijk: 
de bevoegdheid om te subdelegeren moet uit de originele delegatie verwoord zijn: 
``bij of krachtens \ldots''. Ook ``de wet regelt'' staat delegatie toe. ``bij de wet''
of ``de wet bepaalt'' staat dit niet toe. 

\subsection{Bekendmaking van wetten}

Formele wetten en AMvBs worden bekend gemaakt in het \emph{Staatsblad}. Ministeriële 
Regelingen worden gepubliceerd in de \emph{Staatscourant}. 

\subsection{Voorrang van Wetgeving}

\begin{enumerate}
  \item \emph{lex superior derogat legi inferiori}. Een belangrijke uitzondering hierop 
    is artikel 120 Gw, die toetsen aan de grondwet van formele wetgeving verbiedt. 
  \item \emph{lex posterior derogat legi priori}: dit kan op drie manieren:
    \begin{enumerate}
      \item Exclusieve werking
      \item Eerbiedigende werking
      \item Terugwerkende kracht. Alleen de formele wetgever kan hierin afwijken van Wet AB
        die het in principe verbiedt. Bovendien mag het in het strafrecht nooit.
    \end{enumerate}
  \item \emph{lex specialis derogat legi generali}
\end{enumerate}

\subsection{Toetsing van wetgeving}

De Kroon kan besluiten van lagere oerheden vernietigen. Dit kan wegens strijd
met de wet of vanwege strijd met het algemeen belang. De Kroon kan dit
spontaan, dus zonder voorafgaande vraag, doen. In een aantal gevallen moet een
regeling zelfs eerst worden voorgelegd aan provincie of Kroon.

De rechter kan ook wetgeving toetsen, inclusief wetten in formele zin. Hij kan
een wet onverbindend verklaren, maar daarmee houdt die wet niet op te bestaan:
dat geldt alleen tussen de procespartijen. Dat betekent niet dat het verder
geen effect heeft: de rechter pleegt zich aan eerdere uitspraken te houden. 

Zie bijvoorbeeld \textbf{HR Schiermonnikoog} waarin een APV de wegenverkeerswet
doorkruiste.

Als er schade ontstaat door een wettelijke bepaling die in strijd is met een
hogere regel, dan kan er een plicht tot schadevergoeding ontstaan: \textbf{HR
Pocketbooks}.

Onrechtmatige wetgeving kan ook buiten werking gesteld worden als door de
werking van de wet schade ontstaat: zie \textbf{HR Landelijke
Specialistenvereniging}.

Een wet in formele zin mag niet getoetst worden aan algemene rechtsbeginselen
\textbf{HR Harmonisatiewet}.

Lagere wetgeving mag wel getoetst worden aan algemene rechtsbeginselen:
\textbf{HR Landbouwvliegers}. De rechter mag echter niet meer dan de vraag of
``het desbetreffende overheidsorgaan (\ldots) in redelijkheid tot het
desbetreffende voorschrift is kunnen komen'' toetsen, en niet de innerlijke
waarde van de wet beoordelen.

\subsection{Het verdrag als rechtsbron}

Verdragen die een ieder verbindende bepalingen bevatten werken direct
door in het Nederlandse recht (monisme). Dit ligt vast in artikel 93 Gw.

\section{De Rechtspraak}

De handhaving van het recht is in artikel 112 e.v. Gw opgedragen aan de
rechterlijke macht.

De wet op de Rechterlijke Organisatie (RO) bepaalt in artikel 2 dat de 
volgende gerechten tot de rechterlijke macht behoren:

\begin{itemize}
  \item rechtbanken
  \item gerechtshoven
  \item Hoge Raad 
\end{itemize}

Het bestuursorgaan van de rechterlijke macht is de Raad voor de Rechtspraak. 

\subsection{Beginselen van Rechtspraak}

Elk proces dient eerlijk te verlopen (artikel 6 EVRM). 

Hieruit volgen de volgende beginselen:

\begin{enumerate}
  \item De terechtzitting is openbaar (artikel 121 Gw) Uitgezonderd: zaken met
    kinderen en personen- en familierecht. 
  \item De uitspraak is openbaar. Zonder uitzondering.
  \item De rechter is is onafhankelijk. Dit wordt op een aantal manieren gewaarborgd 
    in de Wet RO. Zie p99 Verheugt.
  \item De rechter is onpartijdig. Een rechter kan zich verschonen of gewraakt worden 
    als dit in het geding komt. \footnote{\textbf{HR Onpartijdige Rechter}: mr. D was 
    OvJ in eerste aanleg, later raadsheer (rechter) in beroep. Rechter partijdig verklaard,
    vonnis vernietigd}
  \item De uitspraak is gemotiveerd. (artikel 121 Gw) 
  \item Hoor en wederhoor (bvb artikel 19 Rv) 
  \item Redelijke termijn 
  \item Rechtspraak geschiedt door beroepsrechters. Kan echter wel deelgenomen worden
    door experts (pachtkamer, ondernemingskamer Amsterdam)
\end{enumerate}

\subsection{Het Openbaar Ministerie}

Het OM staat ook wel bekend als de \emph{staande magistratuur}. Haar taken zijn geregeld
in artikel 124 e.v. RO. Het OM heeft het \emph{vervolgingsmonopolie}. Bij elk arondissement
hoort een arondissementsparket. Aan het hoofd van een arondissementsparket staat de hoofdofficier
van justitie. Ook is er een landelijk parket. De vier resorts van de vier gerechtshoven hebben
ieder een resortsparket. Daar werken advocaten-generaal. Een hoofdadvocaat-generaal heeft de
leiding bij zo'n resortparket. Aan het hoofd van OM staat het College van procureurs-generaal,
dat uit drie tot vijf leden bestaat. De politieke verantwoordelijkheid voor het OM ligt bij
de minister van veiligheid en justitie. De minister kan volgens artikel 127 RO algemene
en bijzondere aanwijzingen geven aan het OM. 

\subsection{Rechtspraak in eerste aanleg}

Een zaak die voor het eerst voor een rechter komt wordt ``in eerste aanleg''
genoemd. Dit gebeurt in de regel bij een rechtbank. Men kan vervolgens in
beroep bij het gerechtshof, en de hoge raad kan vervolgens gevraagd worden een
arrest van het gerechtshof te vernietigen (cassatie doen). 

Artikel 42--45 RO bepalen dat de rechtbanken in eerste aanleg kennis nemen van
burgerlijke zaken, bestuurszaken, belastingzaken en strafzaken.

Er zijn enkelvoudige en meervoudige kamers. In meervoudige kamers nemen dire
rechters zitting. Er wordt daar besloten bij meerderheid van stemmen, maar de
stemming wordt nooit bekend gemaakt: dit heet het geheim van de raadkamer
(artikel 7 lid 3 RO).

\subsection{De bevoegheid van de kantonrechter}

De bevoegheid van de kantonrechter in burgerlijke zaken is in artikel 93 en 94 RO
geregeld. Er zijn vier privaatrechtelijke categorieën waarvan de kantonrechter in eerste aanleg kennis
neemt: 

\begin{enumerate}
  \item Zaken met een vordering tot 25000 euro, incl. rente
  \item Zaken met een vordering van onbepaalde waarde, maar waarde $\leq$ 25.000
  \item Zaken betreffende arbeidsovereenkomst, huurovereenkomst, huurkoop
  \item Andere zaken ten aanzien waarvan de wet dit bepaalt.
\end{enumerate}

De kantonrechter is ook bevoegd in bepaalde strafzaken (artikel 382 Sv).

De kantonrechter is in bestuurszaken slechts bevoegd bij Wet Mulder-overtredingen.

\subsection{Relatieve bevoegdheid}

In burgerlijke zaken is relatief bevoegd de rechter van de woonplaats van de
gedaagde. (Artikel 99 lid 1 Rv)

In strafzaken is de rechter bevoegd binnen wiens rechtsgebeid het strafbare
feit is gepleegd. (artikel 2 Sv) 
\\\\
In bestuurszaken: artikel 8:7 Awb:
\begin{description}
  \item[Centrale overheid] Woonplaats belanghebbende
  \item[Bestuursorgaan lagere overheid] Arondissement waar bestuursorgaan zetel heeft
\end{description}

\subsubsection{Terminologie}
\begin{itemize}
  \item De rechter \emph{wijst} vonnis
  \item Het hof / de Hoge Raad wijst arrest
  \item Een rechter geeft beschikking
  \item In bestuursrecht \emph{doet} een rechter uitspraak.
\end{itemize}

\subsection{Rechtspraak in hoger beroep en cassatie}

In burgerlijke zaken wordt beroep ingesteld bij het gerechtshof van het 
ressort waarin het gerecht in eerste aanleg is gevestigd. Tegen een vonnis
van de kantonrechter is geen beroep mogelijk als de waarde van de oorspronkelijke 
vordering $<$ 1750. (artikel 332 Rv) 

Ook in strafzaken doet men beroep bij het gerechtshof. Dat kan niet bij
kantonzaken als de kantonrechter geen straf of maatregel heeft opgelegd, of als
de opgelegde boete lager is dan 50. 

In bestuurszaken is het ingewikkelder: hoofdregel: \emph{Afdeling bestuursrechtspraak
van de Raad van State}. Bij \emph{belastingzaken} is het hoger beroep het gerechtshof,
bij \emph{Wet Mulder-zaken} het gerechtshof in Leeuwarden. Betreft het geschil
\emph{ambtenarenrecht of sociaal zekerheidsrecht}, dan dient het hoger beroep bij de
\emph{Centrale Raad van Beroep}.

\subsubsection{De Hoge Raad der Nederlanden}

In cassatie staan de feiten niet opnieuw ter discussie; enkel kan geklaagd
worden over het niet goed toepassen van het recht door een lagere rechter. 

De Hoge Raad kan door lagere rechters ook prejudiciële vragen gesteld worden
bij problemen met het uitleggen of toepassen van rechtsregels waarover de Hoge
Raad eerder nooit een oordeel heeft gegeven, zie artikel 81a Ro. 

Er zijn twee \emph{cassatiegronden} mogelijk: (artikel 79 Ro)

\begin{description}
  \item[Verzuim van Vormen] Als de rechter zich niet heeft gehouden aan de regels van
    procesrecht kan hierover geklaagd worden.
  \item[Schending van het recht] Klagen over onjuiste toepassing van een regel van 
    materieel recht. 
\end{description}

\subsubsection{De Procureur-Generaal bij de Hoge Raad}

De procureurs-generaal bij de Hoge Raad, of hun plaatsvervangers de
advocaten-generaal bij de Hoge Raad, horen niet bij het OM. Ze zijn aan niemand
ondergeschikt of verantwoording verschuldigd en worden tot het leven benoemd.
Hun belangrijkste taak is conclusie nemen in burgerlijke zaken en strafzaken
(artikel 111 lid 2 sub b RO). Conclusie is slechts een advies aan de HR.

De procureur-generaal kan ook cassatie in belang der wet indienen. Dit gebeurde
bijvoorbeeld in \textbf{HR Doodslag?} waarin een artikel 12 Sv-procedure ook
werd uitgebreid naar oordelen over vervolgingsbeslissingen.

Tot slot kan de procureur-generaal bij de Hoge Raad ontslag van rechterlijke
ambtenaren vorderen bij de Hoge Raad, en kan hij in opdracht van de regering of
Tweede Kamer strafvervolgen tegen ambtsmisdrijven van leden van de
Staten-Generaal, ministers, staatssecretarissen en hoge ambtenaren.

\subsection{Rechtsvorming door rechters}

Rechters moeten zich in het algemeen aan de wet houden, maar moeten soms interpreteren.

Ook houden rechters zich gewoonlijk aan eerdere uitspraken van andere rechters.
De Hoge Raad heeft daarin een belangrijke positie. De Hoge Raad houdt zich
normaal gesproken aan haar eerdere uitspraken, maar soms \emph{gaat} de Hoge
Raad \emph{om}. Een voorbeeld hiervan is \textbf{HR Tongzoen II}: daarin ging
de Hoge Raad om wat betreft de bestempeling van ongewenste tongzoenen als 
verkrachting.

\subsection{Rechtsvinding}



\section{h15} \label{h15}



\end{document}
