\documentclass{article}

\author{Thom Wiggers}
\title{{\Huge Samenvatting}\\\ \\ Inleiding in het Nederlandse Recht, 17e druk, \\ \large J.~W.~P.~Verheugt}

\begin{document}
\maketitle

\section{Recht in het Algemeen}
\label{h1}

\subsection{Rechtsbronnen}

\emph{Positief recht} is het geheel van geldende rechtsregels. Dit staat ook
wel bekend als \emph{Objectief recht}. Dit zijn dus algemeen geldende
bepalingen. Een \emph{subjectief recht} is een bevoegdheid die iemand in een
concreet geval aan een objectief recht ontleent.

\emph{Rechtsbronnen} zijn bronnen waaruit het recht voorkomt. In Nederland
worden de volgende rechtsbronnen onderscheiden:

\begin{itemize}
  \item de wet
  \item jurisprudentie
  \item de gewoonte
  \item verdragen en besluiten van volkenrechtelijke organisaties.
\end{itemize}

Er zijn verschillende onderscheidingen mogelijk tussen de verschillende
onderdelen van het positieve recht. Daarover gaat de rest van dit hoofdstuk.

\subsection{Nationaal en internationaal recht}

\emph{Soevereiniteit} betekent dat elk land in zijn grondgebied de omvang en
inhoud van zijn nationale rechtsstelsel mag bepalen. Hierbij hoeft hij geen
invloeden van buitenaf te dulden.

De relaties tussen landen worden beheersd door een door de eeuwen heen ontstaan
stelsel van normen. Die normen bestaan deels door gewoonte, deels door
verdragen. Dit geheel wordt het \emph{Volkenrecht} genoemd.

\emph{Verdragen} zijn schriftelijke, bindende regelingen tussen staten
onderling of tussen staten en volkenrechtelijke organisaties. Er zijn
verschillende typen verdragen. Zo is een veelvoorkomende soort verdragen
waarbij enkel wederzijdse afspraken gemaakt wordt tussen regeringen,
bijvoorbeeld over politiesamenwerking in grensregio's. Een ander type verdrag
kan de deelnemende staten verplichten bepaalde soorten wetgeving toe te passen
of te maken. Zie hiervoor ook hoofdstuk \ref{h15}. Er zijn ook verdragen die
ingrijpen op de soevereiniteit van landen door directe werking te hebben zonder
tussenkomst van de wetgever. Dit is bijvoorbeeld het EVRM. Nederland heeft een
\emph{monistisch} systeem, dat wil zegen dat rechtsregels uit een verdrag
zonder omzetting in een nationale wet deel kunnen uitmaken van het nationale
recht. Dit is vastgelegd in artikel 93 Gw. Een laatste type verdrag bestaat uit
regelingen waarbij besturende, wetgevende en rechtssprekende bevoegdheid wordt
toegewezen aan supranationale organisaties. Een voorbeeld van zo'n verdrag is
het verdrag betreffende de werking van de Europese Unie. De regering kan
dergelijke inperkingen van de soevereiniteit van Nederland doen overeenkomstig
artikel 92 Gw.

Een regel van een verdrag of een besluit van internationale herkomst gaat boven
een regel van nationaal recht. Zie artikel 94 Gw.

\subsection{Materieel en formeel recht}

\emph{Materieel recht} zijn de regels over de rechten en plichten van personen
in hun onderlinge verkeer. \emph{Formeel recht} zijn de regels over procederen
voor de rechter. Bijvoorbeeld Sr vs Sv.

\subsection{Rechtsgebieden}

\subsubsection{Staatsrecht}

Het staatrecht bevat de regels die betrekking hebben op de organisaties van de
staat en zijn organen en de bevoegdheden van die organen. Ook omvat het de
verhoudingen van de burgers tot de staat en hoe burgers invloed kunnen
uitoefenen op het functioneren van de overheid.

De \emph{Grondwet} is het belangrijkste fundament in het staatsrecht. Deze
bevat eerst een aantal \emph{Grondrechten} zoals het gelijkheidsbeginsel.
Vervolgens beschrijft de grondwet de inrichting van de belangrijkste
overheidsorganen. Ook worden een aantal onderwerpen aangegeven die in andere
wetten moeten worden uitgewerkt. 

\subsubsection{Bestuursrecht}

Bestuursrecht gaat over de rechtsverhouding tussen een burger en de overheid.
Een \emph{beschikking} is een besluit van een bestuursorgaan dat gevolgen
vaststelt voor een specifieke (rechts-)persoon.

\subsubsection{Strafrecht}

Het strafrecht stelt bepaalde zaken strafbaar en draagt de handhaving daarvan
op aan de overheid. In artikel 16 Gw en artikel 1 lid 1 Sr is vastgelegd dat
niets strafbaar kan worden gesteld anders dan door vooraf wettelijke
strafbaarstelling.

\subsubsection{Burgerlijk recht}

Het burgerlijk recht omvat het personen- en familierecht, het
rechtsbersonenrecht en het vermogensrecht. 

\subsection{Publiekrecht en privaatrecht}

Het publiekrecht omvat het staatsrecht, het bestuursrecht (formeel:
bestuursprocesrecht), het strafrecht (formeel: strafprocesrecht), en het
volkenrecht.

Het privaatrecht omvat het personen- en familierecht, het vermogensrecht en het
rechtspersonenrecht. Het formele privaatrecht is het burgerlijk procesrecht.

\subsection{Andere indelingen}

Er kan onderscheid gemaakt worden tussen \emph{geschreven en ongeschreven recht}:
geschreven recht zijn wetten en verdragen, ongeschreven recht zijn
jurisprudentie en de gewoonte. 

Ook wordt het recht wel eens ingedeeld naar functie: bijvoorbeeld
Wegenverkeerswet.

\section{Recht en staat}
\label{h2}

\subsection{Trias Politica}

Montesquieu heeft de \emph{scheiding van machten bedacht}, waarbij de
rechtssprekende, wetgevende en besturende machten (organen) strikt gescheiden
zijn. Aanvullend moeten er \emph{``checks and balances''} zijn: respectievelijk
onderlinge toezichtstaken en gedeelde bevoegdheden. 

Een invloed van Montesquieu is het \emph{legisme}, de filosofie dat de wet,
vastgesteld door de wetgever, de enige rechtsbron is. Een ideaal van het
legisme is de \emph{codificatiegedachte}, de gedachte om al het recht in
systematische wetboeken vast te leggen.

\subsection{De Trias Politica in Nederland}

In Nederland is geen strikte implementatie van de trias politica. De
doelstelling van een machtsevenwicht lijkt echter wel bereikt.

\subsubsection{De wetgevende macht}
 
In Nederland is de wetgevende macht opgedragen aan zowel de Staten-Generaal
als aan de regering, zie artikel 81 Gw. 

\subsubsection{De uitvoerende macht}

De uitvoerende macht is opgedragen aan de regering, die bestaat uit de Koning
en de ministers. (artikel 42 Gw ev) De Staten-Generaal oefent toezicht uit op
de regering. 

Besluiten van de regering heten \emph{Koninklijk Besluit (KB)}. Deze kunnen twee 
vormen hebben: een beschikking of wetgeving. Als een KB wetgeving betreft wordt deze 
ook wel een \emph{Algemene Maatregel van Bestuur (AMvB)} genoemd. De regering 
heeft een speciale wetgevende bevoegdheid volgens artikel 81 Gw.

\subsubsection{De rechtsprekende macht}

De rechtsprekende macht is opgedragen aan de rechterlijke macht en de gerechten 
die niet tot de rechterlijke macht horen, zie artikel 112 e.v. Gw. Het meeste 
over de rechterlijke macht is geregeld in de Wet RO. 

In een bepaald aantal gevallen (bvb wijzigen voornaam) doet de rechter ook 
beschikkingen, maar de rechter mag nooit wetgeving maken: dit wordt door de
Artikel 12 Wet AB uitdrukkelijk verboden.

Ook mag de rechter de wetgever niet bevelen om wetgeving tot stand te brengen:
zie \textbf{HR Nitraatrichtlijn}.

De onafhankelijkheid van de rechter wordt onder meer gegarandeerd doordat de
rechter niet ondergeschikt aan enig overheidsorgaan is. Ook zijn rechters geen
verantwoording verschuldigd aan collega's. 



\section{h15}
\label{h15}

\end{document}
