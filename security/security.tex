\documentclass{article}

\usepackage{amsmath}
\usepackage{amsfonts}
\usepackage{a4wide}

\newcommand{\wmod}[1]{\text{\quad (mod }#1\text{)}}
\newcommand{\Z}{\mathbb{Z}}
\newcommand{\N}{\mathbb{N}}
\newcommand{\egcd}{\operatorname{egcd}}

\author{Thom Wiggers}
\title{Samenvatting Security 2012--2013}

\begin{document}
\maketitle
\section{Security Goals}


\begin{description}
  \item[Confidentiality] Third parties cannot read what $A$ and $B$ are saying.
  \item[Integrity] Third parties cannot modify the contents of the communication.
  \item[Authenticity] $A$ and $B$ are certain about each other's identities.
  \item[Availability] Third parties cannot prevent $A$ and $B$ from communicating.
  \item[Non-repudiation] Someone cannot deny what they have communicated.
  \item[Accountability] Reliale log of communication history.
\end{description}

\section{Crypto primitives}

\begin{tabular}[h]{c||p{100pt}|l|l}
  \bf \# Keys & \bf Name & \bf Key names & \bf Notation \\ \hline \hline
  0 & Hash functions & --- & $h(m)$ \\ \hline
  1 & Symmetric Crypto & Shared, secret & $K\left\{ m \right\}$  \\\hline
  2 & Asymmetric Crypto (public key crypto) & Public \& private keypair & $\left\{ m \right\}_K$ \\
\end{tabular}

\section{Active Attacks}

\subsection{Replay attack}

Intercepted data is sent again. Countermeasures: nonces, timestamps.

\subsection{Reflection attack}

Data from one session is reused in another session, to for example make the
target do encryption/decryption work.

\subsection{Man-in-the-middle attacks}

Also passive: relay attack. Active involves reencryption. countermeasure:
strong authentication.

\section{Cipher modes}

\subsection{Electronic Code Book (ECB)}
$m = m_0m_1m_2\cdots \rightarrow C = K\{m_0\},K\{m_1\},K\{m_2\}\cdots$.

Attack vectors: occurrence frequency, swappping can go unnoticed easily.

\subsection{Cipher Block Chaining (CBC)}

\begin{align*}
  c_0     &= K\{m_0 \oplus IV\} \\
  c_{n+1} &= K\{m_{n+1} \oplus c_n\} \\
\end{align*}

$IV$ may be sent openly or be constant.

One garbled block means two blocks are lost in deciphering. Last block
can be used to verify integrity.

\subsection{Output Feedback Mode (OFB)}

First, pick a random number (Initialisation Vector, $IV$) and use it to
create a keystream:

\[ K\{IV\},K\{K\{IV\}\}, K\{K\{K\{IV\}\}\} \cdots \]

Then XOR with incoming bitstream. Garbled bits are lost, but only those bits.
If sender/receiver are out of sync, everything is lost.

Variation: $c_n = m_n \oplus K\{IV + n\}$ (Counter mode).

\subsection{Cipher Feedback (CFB)}

\begin{align*}
  c_0 &= IV \\
  c_{n+1} &= K\{c_n\} \oplus m_{n+1}
\end{align*}

\section{Symmetric Crypto}

\subsection{Basic Techniques}

\begin{enumerate}

  \item \textbf{Substitution}: Swapping characters from the alphabet. Key $K$
    is the subsitution function.

  \item \textbf{Transposition}: Changing positions of characters (by block).
    $K$ is the position exchange function.

  \item \textbf{One-time pad}: Take bitwise XOR of message with keystream.
    $K$ is keystream of at least the same length as the message.

\end{enumerate}

One-time pads are sometimes generated using linear feedback shift registers. See
slide 25.

A downside of symmetric crypto is that one needs ${N \choose 2}
= \frac{N(N-1)}{2}$ keys if $N$ people want to communicate pairwise securely.

Also, if key $K$ is lost by $A$, $B$ is also affected.

\subsection{Basic Protocols}

\subsubsection{Integrity}

\[
  A \longrightarrow B : m, K_{AB}\{h(m)\}
\]

Hash function for efficiency.

$B$ can verify integity by decrypting and comparing $h(m)$ with the hash he
creates from $m$.

\subsubsection{Confidentiality}

\[ 
  A \longrightarrow B : K_{AB}\left\{ m \right\}
\]

Only those with $K_{AB}$ can read this, obviously.

When combined with integrity ($A \longrightarrow B : K\{m, K\{h(m)\}\}$) it is
important to use different keys, and thus if one is compromised, both integrity
and confidentiality are broken.

\subsubsection{Authenticity}

``Shared Secret'': You can be authenticated by something only you and the other
party know. Problem: secret used in the clear.

It is better to send riddles that can only be solved efficiently using the
secret key. The riddle needs to be fresh every time (against replay attacks).
Often achieved by using \emph{nonces}.

\begin{align*}
  A & \longrightarrow  B  : A, N_A             \\
  B & \longrightarrow  B : K_{AB}\{N_A, N_B\} \\
  A & \longrightarrow  B  : N_B                \\
\end{align*}

\section{Hashing}

\subsection{Properties}

\begin{description}

  \item[Preimage Resistant (one-way)] Given hash value $x$, it should be 
    hard to find $m$ with $h(m) = x$.

  \item[Second preimage resistant] Given an $m$, it should be hard to find 
    $m' \neq m$ with $h(m) = h(m')$.

  \item[Collision resistant] It should be hard to find \emph{any} pair $m \neq
    m'$ with $h(m) = h(m')$.

\end{description}


\subsection{Non-revealing commitment}

e.g. for flipping coins one can use:

\begin{align*}
  A & \longrightarrow B : h(C_A, N_A) \\
  B & \longrightarrow A : h(C_B,N_B)  \\
  A & \longrightarrow B : C_A, N_A    \\
  B & \longrightarrow A : C_B, N_B    \\
\end{align*}

(nonces are used to prevent cheating by lookup table - coin outcomes are very
limited)

\subsubsection{Lamport's hash}

$C$ has for each user $A$ a pair $[n \in \mathbb{N}, h^n(\text{passwd}_A)]$.

\begin{align*}
  A & \rightarrow C : A \\
  C & \rightarrow A : n \\
  A &\rightarrow C  : h^{n-1}(\text{passwd}_A) = x
\end{align*}

$C$ can then verify the authenticity of $A$ by checking $h(x)
= h^n(\text{passwd}_A)$. $C$ can also then set a new pair $[n-1,x]$ or
$[n+1, h(h(x))]$.

\section{Asymmetric Crypto}

Using two keys: one for decryption (private key, $K_d$) and one for encryption
(public key, $K_e$).

\begin{itemize}
  \item Encryption: $\{m\}_{K_e} $
  \item Decryption: $[c]_{K_d}$.
\end{itemize}

Identity function: 
\[
  [\{m\}_{K_e}]_{K_d} = m
\]

Though for some systems $\{[m]_{K_d}\}_{K_e} =m$ is also valid, this is not
a requirement.

\section{Number theory}

\subsection{Equivalence classes}

\[ \forall N \in \N, n \in \Z, m \in \Z 
  \,[n \equiv m \wmod{N} \Leftrightarrow 
      \text{ there is a } k \in \Z \text { with } n = m + k \cdot N]
\]
$\Z_N$ is the set of numbers modulo $N$. Thus 
\[ \Z_N = \{ 0, 1, \dots N-1\} \]
For every $m \in \Z$ we have $m \mod N \in \Z_N$.

If a product modulo $N$ is 1, for instance, $4 \cdot 4 \equiv 1 \wmod{15}$, you
can say $\frac{1}{4}=4 \wmod{15}$. For $\Z_p$ where $p$ is prime, every
non-zero number $n \in \Z_p$ has an inverse $\frac{1}{n} \in \Z_p$.

A finite cyclic group often has a generator $g$ that $g^n = k$ so that $k$ can become
any number in the group.

\subsection{Greatest Common Divisor}

$\gcd(n,m) = $ greatest $k$ which divides both $n$ and $m$. If $\gcd(n,m) = 1$, one
calls $n,m$ relative prime.

\[ 
  \gcd (n,m) = \begin{cases}
    n & m=0 \\
    \gcd(m, n \operatorname{mod} m) & \text{otherwise}\\
  \end{cases}
\]

\subsection{Extended GCD}

\[
  \egcd(m,n) = \begin{cases}
    \langle 1,0 \rangle & n = 0 \\
    \langle y, x - (y \cdot (m \operatorname{div} n))\rangle & \langle x, y \rangle = \egcd(m, n \mod n) \\
  \end{cases}
\]

Results in the solutions $a,b$ for $a \cdot m + b \cdot n = \gcd(m,n)$.

\section{RSA}

RSA has a public key $e$, a private key $d$ and a modulo $n=p\cdot q$. \\
Public key: $(n, e)$\\
Private key: $(n,d)$ \\

\subsection{RSA key creation}
\begin{enumerate}
  \item Pick two primes $p$ and $q$. $n=pq$
  \item Calculate $\phi = (p-1)(q-1)$.
  \item Pick $e$ from $\Z^*_\phi$.
  \item Calculate $d$ from $\frac{1}{e} \in \Z^*_\phi$ using $\egcd(\phi,e)$.
\end{enumerate}

\subsection{Encryption / Decryption}
Encrypt using RSA by $c = m^e \mod n$\\
Decrypt using $m = c^d \mod n$.

\subsection{Signatures with RSA}

\[
  A \longrightarrow B : m, [h(m)]_{d_A}
\]

Works because RSA is symmetric.

\subsubsection{Blind signatures}

\begin{enumerate}
  \item $A$ computes $m' = r^e \cdot m \mod n$, where $r$ is a random number, and gives
    this to $B$.
  \item $B$ signs $m'$, giving $k = [m']^d \mod n$ to $A$.
  \item $A$ computes $\frac{k}{r} = \frac{r^{ed} \cdot m^d}{r} = \frac{r \cdot m^d}{r} = m^d \mod n$
\end{enumerate}

Can be used for anonymous signatures on e-cash, e-voting\dots

\subsection{Certificates}
A certificate is a public key that has been signed by a thrusted third party.

\section{Diffie-Hellman Key Exchange}

Using a generator $g$ in a finite cyclic group:

\begin{align*}
  A &\longrightarrow B: A, g^{s_A} \\
  B &\longrightarrow A: B, g^{s_B} \\
\end{align*}

Then $K_{AB} = g^{s_A s_B} = (g^{s_A})^{s_B} = (g^{s_A})^{s_B}$.

This is not authentication!

\section{El-Gamal}
Fix a generator $g \in G$ of size $N$. \\
Private key: $n \in \mathbb{N}$ with $n < N$. \\
Public key: $h = g^n \in G$.

\subsection{Encryption / Decryption}
\subsubsection{Encryption}

\begin{enumerate}
  \item Represent message as $m \in G$.
  \item Choose a random $r < N$.
  \item $\{m\}_h = (g^r, m \cdot h^r)$.
\end{enumerate}

\subsubsection{Decryption}
\begin{enumerate}
  \item Ciphertext $c = (c_1,c_2)$ with $c_i \in G$.
  \item \[
      [(c_1, c_2)]_n = \frac{c_2}{(c_1)^n} = \frac{m \cdot h^r}{(g^r)^n} = \frac{m \cdot (g^n)^r}{(g^r)^n} = m
    \]
\end{enumerate}


\subsubsection{Signatures}

Choose random $r < p-1$ in $\Z^*_p$ with $\gcd(r,p-1) = 1$ and then:
\[
  \operatorname{sign}_n(m) = \left( g^r, \frac{H(m) - n \cdot g^r}{r} \mod p-1 \right)
\]

Verification of signature $(s_1, s_2)$:
\[
  g^{H(m)} \overset{??}{=} s_1^{s_2}, h^{s_1}
\]


\end{document}
