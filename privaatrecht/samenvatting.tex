\documentclass{article}

\usepackage[utf8]{inputenc}

\usepackage{microtype}

\usepackage[dutch]{babel}

\usepackage{hyperref}

\usepackage[left]{eurosym}

\usepackage{xspace}

\newcommand{\de}{\ensuremath{^\textrm{e}}}

\newcommand{\art}[1]{\textbf{(art.~#1 B.W.)}\xspace}

\author{Thom Wiggers}

\title{{\Huge Samenvatting Privaatrecht}\\\ \\ Op basis van: \\
\emph{Hoofdstukken vermogensrecht, 10\de druk}, \\
{\large J.~H.~Nieuwenhuis} en \emph{Compendium Nederlands Vermogensrecht, 11\de
druk}, {\large{Hijma en Olthof.}}}

\begin{document} \maketitle

\section{Vermogensrechten}

\subsection{Inleiding}

\emph{Vermogensrechten} zijn gedefinieerd in \art{3:6}.

\begin{quote}

  Rechten die, hetzij afzonderlijk hetzij tezamen met een ander recht,
  overdraagbaar zijn, of er toe strekken de rechthebbende stoffelijk voordeel
  te verschaffen, ofwel verkregen zijn in ruil voor verstrekt of in het
  vooruitzicht gesteld stoffelijk voordeel, zijn vermogensrechten. \art{3:6}

\end{quote}

\subsubsection{Goed, zaak, registergoed}

\emph{Goederen} zijn alle zaken en alle vermogensrechten. \art{3:1}

\emph{Zaken} zijn alle voor de menselijke beheersting vatbare stoffen
\art{3:2}.

\emph{Registergoederen} zijn goederen voor welker overdracht of vestiging
inschrijving in de openbare registers noodzakelijk is \art{3:10}. Alle
onroerende zaken zijn registergoederen.

\subsection{Eigendom en vorderingsrecht}

Een \emph{relatief} recht is een rechtsbetrekking tussen bepaalde personen.
Een \emph{absoluut} recht kan worden gehandhaafd jegens iedereen.

\emph{Eigendom} is een exclusief recht, men hoeft geen anderen te dulden, en
heeft gevolg: het blijft op de zaak rusten ook al raakt zij in andere handen.

Een \emph{zakelijk} recht is een recht op een zaak. Een \emph{persoonlijk}
recht is een aanspraak jegens een bepaalde persoon. Eigendom is een zakelijk
recht \art{5:1}, een vorderingsrecht een persoonlijk recht.

Zakelijke rechten zijn onderworpen aan het \emph{individualiseringsprincipe}:
het moet te bepalen zijn precies om welke zaken het gaat. Eigendom op twee
fietsen in een verzameling van fietsen is niet mogelijk, tenzij het om
specifieke exemplaren gaat.

Ook zijn zakelijke rechten onderworpen aan het eenheidsbeginsel: de eigenaar
van een zaak is eigenaar van al haar bestanddelen. \art{5:3} Wat bestanddelen
zijn wordt bepaald in \art{3:4}: verkeersopvatting en een ``hechte fysieke
verbinding''.

\subsubsection{Beperkte rechten}

\begin{quote}
  
  Een beperkt recht is een recht dat is afgeleid uit een meer omvattend recht,
  hetwelk met het beperkte recht is bezwaard. \art{3:8}
  
\end{quote}

Er is slechts een beperkt aantal beperkte rechten: die vallen uiteen in
gebruiksrechten en zekerheidsrechten.

Als er meerdere beperkte rechten zijn gevestigd gaat het oudere beperkte recht
voor.

\subsubsection{Botsingen van vorderingsrechten}

Het gehele vermogen van de schuldenaar staat in voor zijn schulden
\art{3:276}. Zijn schuldeisers worden uit de opbrengst van dit vermogen
voldaan naar evenredigheid van ieders vordering \art{3:277}.

\begin{quote}

  Hebben meerdere schuldeisers ten aanzien van \'e\'en goed met elkaar
  botsende rechten op levering, dan gaat het oudste recht op levering voor,
  tenzij uit de wet, de aard van hun rechten of uit de eisen van redelijkheid
  of billijkheid anders voortvloeit.\art{3:298}

\end{quote}

\section{Vertegenwoordiging} \label{sec:vertegenwoordiging}

\emph{Zie ook H/O Hoofdstuk 4 (nrs. 71 t/m 84, 85).}

\emph{Vertegenwoordiging krachtens volmacht} is het geven van de bevoegdheid
aan iemand om namens jouw een overeenkomst aan te gaan. De tussenpersoon valt
er bij die overeenkomst 'tussenuit'.

Belangrijkste gevallen van vertegenwoordiging: wettelijke vertegenwoordiging,
vertegenwoordiging van een rechtspersoon door haar bestuurders, op grond van
volmacht, of vertegenwoordiging bij zaakwaarneming.

Voor vertegenwoordiging is vereist \emph{vertegenwoordigingskwaliteit}, het
voor de wederpartij duidelijk zijn dat de tussenpersoon handelt ``in naam
van'', en \emph{vertegenwoordigingsbevoegdheid}, het bevoegd zijn van de
vertegenwoordiger.

\emph{Volmacht} is de door eenzijdige rechtshandeling van de volmachtgever
verleende bevoegdheid om hem/haar te vertegenwoordigen. \art{3:60}.

Een volmacht wordt uitgeoefend d.m.v. \art{3:66}: eisen: (a) binnen de grenzen
van zijn bevoegdheid, (b) in naam van de volmachtgever verricht (c)
rechtshandeling.

Het ontbreken van een toereikende volmacht heeft als gevolg dat niet aan de
eisen van \art{3:66} voldaan wordt en dan komt er geen overeenkomst tot stand.
Er komt wel een overeenkomst tot stand bij \emph{gerechtvaardigd vertrouwen
van de wederpartij} \art{6:61 lid 2} of als de achterman de onbevoegd gesloten
overeenkomst \emph{bekrachtigt} \art{6:69}. De tussenpersoon staat in voor het
bestaan van de volmacht en kan dus schadeplichtig zijn \art{3:70}.

Eisen voor een beroep op \art{3:61 lid 2}:
\begin{enumerate}
  \item Rechtshandeling door B in naam van A
  \item Verklaring of gedraging van A
  \item Aanname dat toereikende volmacht was verleend door A, op grond van die
    gewekte schijn.
  \item Redelijkheid van die aanname.
\end{enumerate}


\section{Verbintenissen uit andere bron dan onrechtmatige daad of overeenkomst}

\subsection{Zaakwaarneming}

\emph{Zaakwaarneming} is het willens en wetens en op redelijke grond inlaten
met eens anders belang, zonder de bevoegdheid daartoe aan een rechtshandeling
of een elders in de wet geregelde rechtsverhouding te ontlenen. \art{6:198}.

Een zaakwaarnemer is bevoegd om in naam van de belanghebbende
rechtshandelingen te verrichten \art{3:201}. Zie ook
\ref{sec:vertegenwoordiging}.

Eisen zaakwaarneming:
\begin{enumerate}
  \item Behartiging van eens anders belang
  \item Willens en wetens
  \item Redelijke grond
  \item Noch krachtens rechtshandeling, noch krachtens elders in de wet
    geregelde rechtsverhouding bestaat een bevoegdheid tot de
    belangenbehartiging.
\end{enumerate}

\section{Arresten}

\begin{description}
  \item[Hofland/Hennis] Uitnodiging tot het doen van een aanbod
  \item[Kribbebijter] Vertegenwoordigingskwaliteit hangt af van de wederzijdse
    verklaringen en gedragingen en wat daaraan afgeleid is / mocht worden
    afgeleid.
  \item[Val Dias/Salters] Verklaring of gedraging van pseudo-volmachtgever in
    \art{3:61 lid 2} kan ook een nalating zijn.
\end{description}



\end{document} 

% vim: set ts=4 sw=2 tw=78 formatoptions=t1 et :
