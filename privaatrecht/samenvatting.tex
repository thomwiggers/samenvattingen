\documentclass[a4paper]{article}

\usepackage[utf8]{inputenc}

\usepackage{microtype}

\usepackage[dutch]{babel}

\usepackage{hyperref}

\usepackage[left]{eurosym}

\usepackage{xspace}

\newcommand{\de}{\ensuremath{^\textrm{e}}}

\newcommand{\art}[1]{\textbf{(art.~#1 BW)}\xspace}

\author{Thom Wiggers}

\title{{\Huge Samenvatting Privaatrecht}\\\ \\ Op basis van: \\
\emph{Hoofdstukken vermogensrecht, 10\de druk}, \\
{\large J.~H.~Nieuwenhuis} en \emph{Compendium Nederlands Vermogensrecht, 11\de
druk}, {\large{Hijma en Olthof.}}}

\begin{document} \maketitle

\section{Vermogensrechten}

\subsection{Inleiding}

\emph{Vermogensrechten} zijn gedefinieerd in \art{3:6}.

\begin{quote}

  Rechten die, hetzij afzonderlijk hetzij tezamen met een ander recht,
  overdraagbaar zijn, of er toe strekken de rechthebbende stoffelijk voordeel
  te verschaffen, ofwel verkregen zijn in ruil voor verstrekt of in het
  vooruitzicht gesteld stoffelijk voordeel, zijn vermogensrechten. \art{3:6}

\end{quote}

\subsubsection{Goed, zaak, registergoed}

\emph{Goederen} zijn alle zaken en alle vermogensrechten. \art{3:1}

\emph{Zaken} zijn alle voor de menselijke beheersting vatbare stoffen
\art{3:2}.

\emph{Registergoederen} zijn goederen voor welker overdracht of vestiging
inschrijving in de openbare registers noodzakelijk is \art{3:10}. Alle
onroerende zaken zijn registergoederen.

\subsection{Eigendom en vorderingsrecht}

Een \emph{relatief} recht is een rechtsbetrekking tussen bepaalde personen.
Een \emph{absoluut} recht kan worden gehandhaafd jegens iedereen.

\emph{Eigendom} is een exclusief recht, men hoeft geen anderen te dulden, en
heeft gevolg: het blijft op de zaak rusten ook al raakt zij in andere handen.

Een \emph{zakelijk} recht is een recht op een zaak. Een \emph{persoonlijk}
recht is een aanspraak jegens een bepaalde persoon. Eigendom is een zakelijk
recht \art{5:1}, een vorderingsrecht een persoonlijk recht.

Zakelijke rechten zijn onderworpen aan het \emph{individualiseringsprincipe}:
het moet te bepalen zijn precies om welke zaken het gaat. Eigendom op twee
fietsen in een verzameling van fietsen is niet mogelijk, tenzij het om
specifieke exemplaren gaat.

Ook zijn zakelijke rechten onderworpen aan het eenheidsbeginsel: de eigenaar
van een zaak is eigenaar van al haar bestanddelen. \art{5:3} Wat bestanddelen
zijn wordt bepaald in \art{3:4}: verkeersopvatting en een ``hechte fysieke
verbinding''.

\subsubsection{Beperkte rechten}

\begin{quote}

  Een beperkt recht is een recht dat is afgeleid uit een meer omvattend recht,
  hetwelk met het beperkte recht is bezwaard. \art{3:8}

\end{quote}

Er is slechts een beperkt aantal beperkte rechten: die vallen uiteen in
gebruiksrechten en zekerheidsrechten.

Als er meerdere beperkte rechten zijn gevestigd gaat het oudere beperkte recht
voor.

\subsubsection{Botsingen van vorderingsrechten}

Het gehele vermogen van de schuldenaar staat in voor zijn schulden
\art{3:276}. Zijn schuldeisers worden uit de opbrengst van dit vermogen
voldaan naar evenredigheid van ieders vordering \art{3:277}.

\begin{quote}

  Hebben meerdere schuldeisers ten aanzien van \'e\'en goed met elkaar
  botsende rechten op levering, dan gaat het oudste recht op levering voor,
  tenzij uit de wet, de aard van hun rechten of uit de eisen van redelijkheid
  of billijkheid anders voortvloeit.\art{3:298}

\end{quote}

\section{Vertegenwoordiging} \label{sec:vertegenwoordiging}

\emph{Zie ook H/O Hoofdstuk 4 (nrs. 71 t/m 84, 85).}

\emph{Vertegenwoordiging krachtens volmacht} is het geven van de bevoegdheid
aan iemand om namens jouw een overeenkomst aan te gaan. De tussenpersoon valt
er bij die overeenkomst 'tussenuit'.

Belangrijkste gevallen van vertegenwoordiging: wettelijke vertegenwoordiging,
vertegenwoordiging van een rechtspersoon door haar bestuurders, op grond van
volmacht, of vertegenwoordiging bij zaakwaarneming.

Voor vertegenwoordiging is vereist \emph{vertegenwoordigingskwaliteit}, het
voor de wederpartij duidelijk zijn dat de tussenpersoon handelt ``in naam
van'', en \emph{vertegenwoordigingsbevoegdheid}, het bevoegd zijn van de
vertegenwoordiger.

\emph{Volmacht} is de door eenzijdige rechtshandeling van de volmachtgever
verleende bevoegdheid om hem/haar te vertegenwoordigen. \art{3:60}.

Een volmacht wordt uitgeoefend d.m.v. \art{3:66}: eisen: (a) binnen de grenzen
van zijn bevoegdheid, (b) in naam van de volmachtgever verricht (c)
rechtshandeling.

Het ontbreken van een toereikende volmacht heeft als gevolg dat niet aan de
eisen van \art{3:66} voldaan wordt en dan komt er geen overeenkomst tot stand.
Er komt wel een overeenkomst tot stand bij \emph{gerechtvaardigd vertrouwen
van de wederpartij} \art{6:61 lid 2} of als de achterman de onbevoegd gesloten
overeenkomst \emph{bekrachtigt} \art{6:69}. De tussenpersoon staat in voor het
bestaan van de volmacht en kan dus schadeplichtig zijn \art{3:70}.

Eisen voor een beroep op \art{3:61 lid 2}:
\begin{enumerate}
  \item Rechtshandeling door B in naam van A
  \item Verklaring of gedraging van A
  \item Aanname dat toereikende volmacht was verleend door A, op grond van die
    gewekte schijn.
  \item Redelijkheid van die aanname.
\end{enumerate}

\section{Het vaststellen van de inhoud van de overeenkomst}

Naast de inhoud van een overeenkomst heeft het contract de rechtsgevolgen die
voortvloeien uit de wet, de gewoonte en de eisen van redelijkheid en
billijkheid. \art{6:248}

De wet heeft meestal een \emph{aanvullend} karakter, voor zover contractanten
niets anders overeenkomen. Soms echter stelt de wet grenzen aan de
contractvrijheid door middel van \emph{dwingend} recht, daar mag niet van
worden afgeweken.

De inhoud van een overeenkomst wordt mede bepaald door wat gewoon is: als iets
gewoon is, dan mag dat ook van de contractpartijen verlangd worden, al staat
dat niet in de overeenkomst.

De redelijkheid en billijkheid hebben een aanvullende werking: zo is een
vennoot verplicht zich te onthouden van concurrentie jegens zijn medevennoten,
ook al vermeldt het contract daarover niets. Redelijkheid en billijkheid
kunnen ook beperkend zijn, zie \art{6:248 lid 2}.

Bij onvoorziene omstandigheden kan de rechter op verlangen van een der
partijen de overeenkomst wijzigingen als door de omstandigheden naar
maatstaven van redelijkheid en billijkheid instandhouding niet mag worden
verwacht. \art{6:258}

\subsection{Rechtsgevolgen voor derden}

Hoofdregel: overeenkomsten zijn alleen van kracht tussen partijen.
Hierop zijn wettelijke en buitenwettelijke uitzonderingen.

Wettelijke uitzonderingen:
\begin{enumerate}
  \item Kwalitatieve rechten \art{6:251}
  \item Kwalitatieve verplichtingen \art{6:252}
  \item Derdenbeding \art{6:253-2:256}
  \item (Blokkering ``paardensprong'' \art{6:257})
\end{enumerate}

\subsubsection{Kwalitatieve rechten}

Rechten gaan mee met een aan de schuldeiser toebehorend goed, indien ze
kwalitatief zijn en overdraagbaar.

\subsubsection{Kwalitatieve verplichtingen}

(1) Verplichtingen tot dulden of niets doen ten aanzien van een (2)
registergoed kunnen overgaan op op diegenen die het goed verkrijgen, als dit
is (3) overeengekomen en (4) vastgelegd in een notariële akte, ingeschreven in
de openbare registers. \art{6:252}

\subsubsection{Alternatieven}

Erfdienstbaarheid \art{5:70} is een alternatief.

Een \emph{kettingbeding} kan ook verplichtingen tot doen bevatten, maar is
niet absoluut afdwingbaar (levert slechts wanprestatie op).

\subsubsection{Derdenbeding}

Er kan d.m.v. derdenbeding een derde het recht een prestatie te vorderen
worden gegeven, als dit in een overeenkomst wordt opgenomen en de derde dat
aanvaardt. Na aanvaarding wordt de derde partij bij de overeenkomst.



\section{Nakoming}

Bij het aangaan van een overeenkomst ontstaan verbintenissen. In ieder van
deze verbintenissen is er een schuldeiser en een schuldenaar. Een verbintenis
beoogt in de regel haar eigen ondergang: zodra de verplichting van de
verbintenis voldaan is, gaat de verbintenis teniet.

\subsection{Betaling}

Betaling is het voldoen van een schuld; dat hoeft niet per se geld te
betekenen. Een overeenkomst tot levering van een auto wordt ook 'betaald' door
de auto te leveren.

\begin{quote}

  Een verbintenis kan ook door een ander dan de schuldenaar worden nagekomen,
  tenzij haar inhoud of strekking zich daartegen verzet. \art{6:30 lid 1}

\end{quote}

Wie de schuldenaar tot betaling kunnen dwingen zijn de schuldeiser (indien
handelingsbekwaam zijn vertegenwoordiger) of diegene die daartoe een volmacht
heeft gekregen van de schuldeiser.

De schuldenaar kan echter aan meer mensen bevrijdend betalen. Wanneer een
handelingsonbekwame (dus onbevoegde) een bedrag ontvangt in plaats van zijn
vertegenwoordiger, en het ontvangene strekt tot zijn \emph{werkelijk
voordeel} of is in de macht van de vertegenwoordiger gekomen, dan is er
bevrijdend betaald. \art{6:31} Met ``tot werkelijk voordeel'' wordt bedoeld
als het ontvangene is ``besteed ter bekostiging van levensonderhoud of studie
of van hetgeen verder voor zijn geestelijk of lichamelijk welzijn dienstig kon
zijn.'' Degene aan wie had moeten worden betaald kan de betaling aan een
onbevoegde ook \emph{bekrachtigen} \art{6:32}. Ook als de rechthebbende
\emph{gebaat} is, is er bevrijdend betaald.

Wanneer de schuldenaar op redelijke gronden dacht dat iemand was, die
dat niet was, kan hij zich beroepen op \art{6:34}.

\subsection{Opschorting}

\begin{quote}

  Een schuldenaar die een opeisbare vordering heeft op zijn schuldeiser, is
  bevoegd de nakoming van zijn verbintenis op te schorten tot voldoening van
  zijn vordering plaatsvindt, indien tussen vordering en verbintenis voldoende
  samenhang bestaat om deze opschorting te rechtvaardigen. \art{6:52}

\end{quote}

Eisen:
\begin{enumerate}
  \item Verbintenis van schuldenaar jegens schuldeiser
  \item Opeisbare vordering van schuldenaar op schuldeiser
  \item Onbevoegd niet-nakomen door schuldeiser
  \item Samenhang tussen vordering en verbintenis:
    \begin{itemize}
      \item Voortvloeien uit dezelfde rechtsverhouding
      \item voortvloeien uit zaken die partijen regelmatig met elkaar gedaan
        hebben.
    \end{itemize}
\end{enumerate}

Voor wederkerige overeenkomsten \art{6:261} is er een speciaal artikel voor
opschorting. \emph{Exceptio non adimpleti contractus}:

\begin{quote}

  Komt een der partijen haar verbintenis niet na, dan is de wederpartij
  bevoegd de nakoming van haar daartegenover staande verplichtingen op te
  schorten. \art{6:262 lid 1}

\end{quote}

Eisen:
\begin{enumerate}
  \item Tegenover elkaar staande verbintenissen
  \item Tekortkoming moet opschorting rechtvaardigen
  \item Eisen van onbevoegdheid en opeisbaarheid van \art{6:52}.
\end{enumerate}

\subsection{Retentierecht}

\art{3:290}

\begin{enumerate}
  \item Verbintenis tot afgifte van een zaak
  \item Verder zelfde eisen als \art{6:52}
\end{enumerate}

\section{Rechten van de schuldeiser bij niet-nakoming door de schuldenaar}

Wanneer een schuldenaar niet nakomt, heeft de schuldeiser een groot aantal
instrumenten ter beschikking. Het is onder meer mogelijk om \emph{nakoming te
vorderen}, \emph{schadevergoeding} te eisen, \emph{op te schorten} of te
\emph{ontbinden}.

\subsection{Nakoming}

Wanneer iemand niet nakomt, kan de rechter tot nakoming veroordelen. Komt de
schuldenaar dan nog niet over de brug, dan wordt de deurwaarder ingeschakeld.
Die \emph{re\"ele executie} vervangt dan vrijwillige nakoming \art{3:297}. Is voor
levering een notari\"ele akte vereist, dan kan de rechter zijn vonnis dezelfde
kracht geven \art{3:300}

\subsection{Schadevergoeding}

Toerekenbare tekortkoming in de nakoming kan leiden tot schadevergoeding
\art{6:74}:

Eisen:
\begin{enumerate}
  \item Tekortkoming in de nakoming
    \begin{enumerate}
      \item Verzuim \art{6:81} benodigd als nakoming nog mogelijk is
      \item Tekortkoming wanneer niet, te laat of ondeugdelijk nakomen.
    \end{enumerate}
  \item Toerekenbaarheid van de tekortkoming (bewijslast bij schuldenaar)
    \begin{enumerate}
      \item Verwijtbare schuld
      \item Risico op basis van:
        \begin{enumerate}
          \item Wet (hulppersonen, etc)
          \item Rechtshandeling (deel overeenkomst)
          \item Verkeersopvatting
        \end{enumerate}
    \end{enumerate}
  \item Schade
  \item Causaal verband: \emph{conditio sine qua non}
\end{enumerate}

\subsection{Toerekenbaarheid van de tekortkoming}

Een tekortkoming is toerekenbaar als gevolg van \emph{schuld} of
\emph{risico}. Zie hierboven.

Wanneer een schuldenaar door niet-toerekenbare tekortkoming verreikt wordt,
dan kan de schuldeiser het voordeel als schade vorderen \art{6:78}.

\subsection{Verzuim}

Verzuim treedt in door een ingebrekestelling \art{6:82}, tenzij de termijn
voor nakoming is verstreken of wanneer de schuldeiser mededeelt dat deze in
nakoming tekort zal schieten \art{6:83}.

Eisen verzuim:
\begin{enumerate}
  \item Nakoming is \emph{niet} blijvend onmogelijk, dan kan immers zonder
    verzuim \art{6:74} toegepast worden.
  \item Tekortkoming van de debiteur:
    \begin{enumerate}
      \item Opeisbare verbintenis
      \item Prestatie blijft uit of is ondeugdelijk
      \item Uitblijven prestatie is niet gerechtvaardigd door een
        opschortingsbevoegdheid
      \item Er is aan de eisen van \art{6:82-83} voldaan
    \end{enumerate}
  \item Toerekenbaarhied van de vertraging aan de debiteur.
\end{enumerate}

Voor het intreden van verzuim is in de regel een \emph{ingebrekestelling}
vereist. Hierin stelt de schuldeiser de schuldenaar een redelijke termijn voor
de nakoming \art{6:82 lid 1}. Als die termijn is verstreken dan raakt de
schuldenaar \emph{in verzuim}, met als gevolg onder meer een
schadevergoedingsplicht en een risico-omslag \emph{6:84}.

Er is geen termijn nodig wanneer de debiteur tijdelijk niet kan nakomen of uit
zijn houding blijkt dat aanmaning nutteloos zou zijn. \art{6:82 lid 2}

Het verzuim is afgelopen als de schuldenaar alsnog nakomt, met
schadevergoeding \art{6:86}. De schuldeiser kan ook de verbintenis tot
nakoming omzetten in een verbintenis tot vervangende schadevergoeding.
\art{6:86}.

\subsection{Ontbinding}

Wanneer de ene partij een verbintenis uit een wederkerige overeenkomst niet
nakomt, is de andere partij bevoegd de overeenkomst te ontbinden. \art{6:265}

De ontbinding kan zowel door een schriftelijke verklaring als door de rechter
worden uitgesproken \art{6:267}. Is de schuldenaar niet verhinderd na te
komen, dan is vereist dat de schuldenaar in verzuim is. \art{6:265 lid 2}

Ontbinding heeft \emph{geen terugwerkende kracht} \art{6:269}. Er ontstaat een
verbintenis tot ongedaanmaking van wat reeds is nagekomen.

\section{Verbintenissen uit andere bron dan onrechtmatige daad of overeenkomst}

\subsection{Zaakwaarneming}

\emph{Zaakwaarneming} is het willens en wetens en op redelijke grond inlaten
met eens anders belang, zonder de bevoegdheid daartoe aan een rechtshandeling
of een elders in de wet geregelde rechtsverhouding te ontlenen. \art{6:198}.

Een zaakwaarnemer is bevoegd om in naam van de belanghebbende
rechtshandelingen te verrichten \art{3:201}. Zie ook
\ref{sec:vertegenwoordiging}.

Eisen zaakwaarneming:
\begin{enumerate}
  \item Behartiging van eens anders belang
  \item Willens en wetens
  \item Redelijke grond
  \item Noch krachtens rechtshandeling, noch krachtens elders in de wet
    geregelde rechtsverhouding bestaat een bevoegdheid tot de
    belangenbehartiging.
\end{enumerate}

\section{Onrechtmatige daad}

\begin{quote}

  Hij die jegens een ander een onrechtmatige daad pleegt, welke hem kan worden
  toegerekend, is verplicht de schade die de ander dientengevolge lijdt, te
  vergoeden. \art{6:162 lid 1}

\end{quote}

Onrechtmatige daad:
\begin{enumerate}
  \item Onrechtmatigheid
    \begin{enumerate}
      \item Inbreuk op een recht
      \item Strijd met een wettelijke plicht
      \item Strijd met hetgeen volgens ongeschreven wet in het
        maatschappelijke verkeer betaamt. Bij gevaarzetting:
        \textbf{Kelderluik}:
        \begin{enumerate} \label{list:kelderluik}
          \item Mate van waarschijnlijkheid waarmee de niet-inachtneming van
            de vereiste oplettendheid en voorzichtigheid kan worden verwacht.
          \item De hoegrootheid van de kans dat daaruit ongevallen ontstaan
          \item De ernst van de gevolgen van een ongeval
          \item De mate van bezwaarlijkheid van te nemen
            veiligheidsmaatregelen.
        \end{enumerate}
    \end{enumerate}
  \item Toerekenbaarheid van de onrechtmatige daad aan de dader.
    \begin{itemize}
      \item Schuld
      \item Toerekenbaarheid krachtens wet of verkeersopvattingen
    \end{itemize}
  \item Schade
  \item Causaal verband tussen schade en daad
  \item Verband tussen het doel van de overtreden norm en het geschade belang:
    het relativiteitsbeginsel
\end{enumerate}

Een toerekenbare onrechtmatige daad wordt ook wel een \emph{fout} genoemd. Een
\emph{rechtvaardigingsgrond} neemt de onrechtmatigheid weg.

Rechtvaardigingsgronden:
\begin{itemize}
  \item Overmacht
  \item Noodweer
  \item Wettelijk voorschrift
  \item Ambtelijk bevel
  \item Toestemming van de benadeelde
\end{itemize}

\section{Arresten}

\begin{description}

  \item[Hofland/Hennis] Uitnodiging tot het doen van een aanbod

  \item[Kelderluik] Zie Eisen onrechtmatige daad, \ref{list:kelderluik}.

  \item[Kribbebijter] Vertegenwoordigingskwaliteit hangt af van de wederzijdse
    verklaringen en gedragingen en wat daaraan afgeleid is / mocht worden
    afgeleid.

  \item[Tandarts] Inbreuk op een ongeschreven verkeersnorm kan ook
    onrechtmatig zijn.

  \item[Val Dias/Salters] Verklaring of gedraging van pseudo-volmachtgever in
    \art{3:61 lid 2} kan ook een nalating zijn.

  \item[Haviltex] Bij de uitleg van de bepalingen in een contract komt het ook
    neer op ``de zin die pp.~in de gegeven omstandigheden over en weer
    redelijkerwijs aan deze bepalingen mochten toekenen en op hetgeen zij te
    dien aanzien redelijkerwijs van elkaar mochten verwachten. Daarbij kan
    mede van belang zijn tot welke maatschappelijke kringen pp.~behoren en
    welke rechtskennis van zodanige pp.~kan worden verwacht.''

\end{description}



\end{document}

% vim: set ts=4 sw=2 tw=78 formatoptions=t1 et :
