\documentclass[a4paper]{article}

\usepackage[utf8]{inputenc}

\usepackage{microtype}

\usepackage[dutch]{babel}

\usepackage{hyperref}

\usepackage[left]{eurosym}

\usepackage{xspace}

\newcommand{\de}{\ensuremath{^\textrm{e}}}

\newcommand{\art}[1]{\textbf{(art.~#1 BW)}\xspace}

\author{Thom Wiggers}

\title{{\Huge Samenvatting Privaatrecht}\\\ \\ Op basis van: \\
\emph{Hoofdstukken vermogensrecht, 10\de druk}, \\
{\large J.~H.~Nieuwenhuis} en \emph{Compendium Nederlands Vermogensrecht, 11\de
druk}, {\large{Hijma en Olthof.}}}

\begin{document} \maketitle

\section{Vermogensrechten}

\subsection{Inleiding}

\emph{Vermogensrechten} zijn gedefinieerd in \art{3:6}.

\begin{quote}

  Rechten die, hetzij afzonderlijk hetzij tezamen met een ander recht,
  overdraagbaar zijn, of er toe strekken de rechthebbende stoffelijk voordeel
  te verschaffen, ofwel verkregen zijn in ruil voor verstrekt of in het
  vooruitzicht gesteld stoffelijk voordeel, zijn vermogensrechten. \art{3:6}

\end{quote}

\subsubsection{Goed, zaak, registergoed}

\emph{Goederen} zijn alle zaken en alle vermogensrechten. \art{3:1}

\emph{Zaken} zijn alle voor de menselijke beheersting vatbare stoffen
\art{3:2}.

\emph{Registergoederen} zijn goederen voor welker overdracht of vestiging
inschrijving in de openbare registers noodzakelijk is \art{3:10}. Alle
onroerende zaken zijn registergoederen.

\subsection{Eigendom en vorderingsrecht}

Een \emph{relatief} recht is een rechtsbetrekking tussen bepaalde personen.
Een \emph{absoluut} recht kan worden gehandhaafd jegens iedereen.

\emph{Eigendom} is een exclusief recht, men hoeft geen anderen te dulden, en
heeft gevolg: het blijft op de zaak rusten ook al raakt zij in andere handen.

Een \emph{zakelijk} recht is een recht op een zaak. Een \emph{persoonlijk}
recht is een aanspraak jegens een bepaalde persoon. Eigendom is een zakelijk
recht \art{5:1}, een vorderingsrecht een persoonlijk recht.

Zakelijke rechten zijn onderworpen aan het \emph{individualiseringsprincipe}:
het moet te bepalen zijn precies om welke zaken het gaat. Eigendom op twee
fietsen in een verzameling van fietsen is niet mogelijk, tenzij het om
specifieke exemplaren gaat.

Ook zijn zakelijke rechten onderworpen aan het eenheidsbeginsel: de eigenaar
van een zaak is eigenaar van al haar bestanddelen. \art{5:3} Wat bestanddelen
zijn wordt bepaald in \art{3:4}: verkeersopvatting en een ``hechte fysieke
verbinding''.

\subsubsection{Beperkte rechten}

\begin{quote}

  Een beperkt recht is een recht dat is afgeleid uit een meer omvattend recht,
  hetwelk met het beperkte recht is bezwaard. \art{3:8}

\end{quote}

Er is slechts een beperkt aantal beperkte rechten: die vallen uiteen in
gebruiksrechten en zekerheidsrechten.

Als er meerdere beperkte rechten zijn gevestigd gaat het oudere beperkte recht
voor.

\subsubsection{Botsingen van vorderingsrechten}

Het gehele vermogen van de schuldenaar staat in voor zijn schulden
\art{3:276}. Zijn schuldeisers worden uit de opbrengst van dit vermogen
voldaan naar evenredigheid van ieders vordering \art{3:277}.

\begin{quote}

  Hebben meerdere schuldeisers ten aanzien van \'e\'en goed met elkaar
  botsende rechten op levering, dan gaat het oudste recht op levering voor,
  tenzij uit de wet, de aard van hun rechten of uit de eisen van redelijkheid
  of billijkheid anders voortvloeit.\art{3:298}

\end{quote}

\section{Vertegenwoordiging} \label{sec:vertegenwoordiging}

\emph{Zie ook H/O Hoofdstuk 4 (nrs. 71 t/m 84, 85).}

\emph{Vertegenwoordiging krachtens volmacht} is het geven van de bevoegdheid
aan iemand om namens jouw een overeenkomst aan te gaan. De tussenpersoon valt
er bij die overeenkomst 'tussenuit'.

Belangrijkste gevallen van vertegenwoordiging: wettelijke vertegenwoordiging,
vertegenwoordiging van een rechtspersoon door haar bestuurders, op grond van
volmacht, of vertegenwoordiging bij zaakwaarneming.

Voor vertegenwoordiging is vereist \emph{vertegenwoordigingskwaliteit}, het
voor de wederpartij duidelijk zijn dat de tussenpersoon handelt ``in naam
van'', en \emph{vertegenwoordigingsbevoegdheid}, het bevoegd zijn van de
vertegenwoordiger.

\emph{Volmacht} is de door eenzijdige rechtshandeling van de volmachtgever
verleende bevoegdheid om hem/haar te vertegenwoordigen. \art{3:60}.

Een volmacht wordt uitgeoefend d.m.v. \art{3:66}: eisen: (a) binnen de grenzen
van zijn bevoegdheid, (b) in naam van de volmachtgever verricht (c)
rechtshandeling.

Het ontbreken van een toereikende volmacht heeft als gevolg dat niet aan de
eisen van \art{3:66} voldaan wordt en dan komt er geen overeenkomst tot stand.
Er komt wel een overeenkomst tot stand bij \emph{gerechtvaardigd vertrouwen
van de wederpartij} \art{6:61 lid 2} of als de achterman de onbevoegd gesloten
overeenkomst \emph{bekrachtigt} \art{6:69}. De tussenpersoon staat in voor het
bestaan van de volmacht en kan dus schadeplichtig zijn \art{3:70}.

Eisen voor een beroep op \art{3:61 lid 2}:
\begin{enumerate}
  \item Rechtshandeling door B in naam van A
  \item Verklaring of gedraging van A
  \item Aanname dat toereikende volmacht was verleend door A, op grond van die
    gewekte schijn.
  \item Redelijkheid van die aanname.
\end{enumerate}

\section{Het vaststellen van de inhoud van de overeenkomst}

Naast de inhoud van een overeenkomst heeft het contract de rechtsgevolgen die
voortvloeien uit de wet, de gewoonte en de eisen van redelijkheid en
billijkheid. \art{6:248}

De wet heeft meestal een \emph{aanvullend} karakter, voor zover contractanten
niets anders overeenkomen. Soms echter stelt de wet grenzen aan de
contractvrijheid door middel van \emph{dwingend} recht, daar mag niet van
worden afgeweken.

De inhoud van een overeenkomst wordt mede bepaald door wat gewoon is: als iets
gewoon is, dan mag dat ook van de contractpartijen verlangd worden, al staat
dat niet in de overeenkomst.

De redelijkheid en billijkheid hebben een aanvullende werking: zo is een
vennoot verplicht zich te onthouden van concurrentie jegens zijn medevennoten,
ook al vermeldt het contract daarover niets. Redelijkheid en billijkheid
kunnen ook beperkend zijn, zie \art{6:248 lid 2}.

Bij onvoorziene omstandigheden kan de rechter op verlangen van een der
partijen de overeenkomst wijzigingen als door de omstandigheden naar
maatstaven van redelijkheid en billijkheid instandhouding niet mag worden
verwacht. \art{6:258}

\subsection{Rechtsgevolgen voor derden}

Hoofdregel: overeenkomsten zijn alleen van kracht tussen partijen.
Hierop zijn wettelijke en buitenwettelijke uitzonderingen.

Wettelijke uitzonderingen:
\begin{enumerate}
  \item Kwalitatieve rechten \art{6:251}
  \item Kwalitatieve verplichtingen \art{6:252}
  \item Derdenbeding \art{6:253-2:256}
  \item (Blokkering ``paardensprong'' \art{6:257})
\end{enumerate}

\subsubsection{Kwalitatieve rechten}

Rechten gaan mee met een aan de schuldeiser toebehorend goed, indien ze
kwalitatief zijn en overdraagbaar.

\subsubsection{Kwalitatieve verplichtingen}

(1) Verplichtingen tot dulden of niets doen ten aanzien van een (2)
registergoed kunnen overgaan op op diegenen die het goed verkrijgen, als dit
is (3) overeengekomen en (4) vastgelegd in een notariële akte, ingeschreven in
de openbare registers. \art{6:252}

\subsubsection{Alternatieven}

Erfdienstbaarheid \art{5:70} is een alternatief.

Een \emph{kettingbeding} kan ook verplichtingen tot doen bevatten, maar is
niet absoluut afdwingbaar (levert slechts wanprestatie op).

\subsubsection{Derdenbeding}

Er kan d.m.v. derdenbeding een derde het recht een prestatie te vorderen
worden gegeven, als dit in een overeenkomst wordt opgenomen en de derde dat
aanvaardt. Na aanvaarding wordt de derde partij bij de overeenkomst.



\section{Nakoming}

Bij het aangaan van een overeenkomst ontstaan verbintenissen. In ieder van
deze verbintenissen is er een schuldeiser en een schuldenaar. Een verbintenis
beoogt in de regel haar eigen ondergang: zodra de verplichting van de
verbintenis voldaan is, gaat de verbintenis teniet.

\subsection{Betaling}

Betaling is het voldoen van een schuld; dat hoeft niet per se geld te
betekenen. Een overeenkomst tot levering van een auto wordt ook 'betaald' door
de auto te leveren.

\begin{quote}

  Een verbintenis kan ook door een ander dan de schuldenaar worden nagekomen,
  tenzij haar inhoud of strekking zich daartegen verzet. \art{6:30 lid 1}

\end{quote}

Wie de schuldenaar tot betaling kunnen dwingen zijn de schuldeiser (indien
handelingsbekwaam zijn vertegenwoordiger) of diegene die daartoe een volmacht
heeft gekregen van de schuldeiser.

De schuldenaar kan echter aan meer mensen bevrijdend betalen. Wanneer een
handelingsonbekwame (dus onbevoegde) een bedrag ontvangt in plaats van zijn
vertegenwoordiger, en het ontvangene strekt tot zijn \emph{werkelijk
voordeel} of is in de macht van de vertegenwoordiger gekomen, dan is er
bevrijdend betaald. \art{6:31} Met ``tot werkelijk voordeel'' wordt bedoeld
als het ontvangene is ``besteed ter bekostiging van levensonderhoud of studie
of van hetgeen verder voor zijn geestelijk of lichamelijk welzijn dienstig kon
zijn.'' Degene aan wie had moeten worden betaald kan de betaling aan een
onbevoegde ook \emph{bekrachtigen} \art{6:32}. Ook als de rechthebbende
\emph{gebaat} is, is er bevrijdend betaald.

Wanneer de schuldenaar op redelijke gronden dacht dat iemand was, die
dat niet was, kan hij zich beroepen op \art{6:34}.

\subsection{Opschorting}

\begin{quote}

  Een schuldenaar die een opeisbare vordering heeft op zijn schuldeiser, is
  bevoegd de nakoming van zijn verbintenis op te schorten tot voldoening van
  zijn vordering plaatsvindt, indien tussen vordering en verbintenis voldoende
  samenhang bestaat om deze opschorting te rechtvaardigen. \art{6:52}

\end{quote}

Eisen:
\begin{enumerate}
  \item Verbintenis van schuldenaar jegens schuldeiser
  \item Opeisbare vordering van schuldenaar op schuldeiser
  \item Onbevoegd niet-nakomen door schuldeiser
  \item Samenhang tussen vordering en verbintenis:
    \begin{itemize}
      \item Voortvloeien uit dezelfde rechtsverhouding
      \item voortvloeien uit zaken die partijen regelmatig met elkaar gedaan
        hebben.
    \end{itemize}
\end{enumerate}

Voor wederkerige overeenkomsten \art{6:261} is er een speciaal artikel voor
opschorting. \emph{Exceptio non adimpleti contractus}:

\begin{quote}

  Komt een der partijen haar verbintenis niet na, dan is de wederpartij
  bevoegd de nakoming van haar daartegenover staande verplichtingen op te
  schorten. \art{6:262 lid 1}

\end{quote}

Eisen:
\begin{enumerate}
  \item Tegenover elkaar staande verbintenissen
  \item Tekortkoming moet opschorting rechtvaardigen
  \item Eisen van onbevoegdheid en opeisbaarheid van \art{6:52}.
\end{enumerate}

\subsection{Retentierecht}

\art{3:290}. \art{6:52} is ook van toepassing via schakelbepaling \art{6:57}.

\begin{enumerate}
  \item Verbintenis tot afgifte van een zaak
  \item Verder zelfde eisen als \art{6:52}
\end{enumerate}

\section{Rechten van de schuldeiser bij niet-nakoming door de schuldenaar}

Wanneer een schuldenaar niet nakomt, heeft de schuldeiser een groot aantal
instrumenten ter beschikking. Het is onder meer mogelijk om \emph{nakoming te
vorderen}, \emph{schadevergoeding} te eisen, \emph{op te schorten} of te
\emph{ontbinden}.

\subsection{Nakoming}

Wanneer iemand niet nakomt, kan de rechter tot nakoming veroordelen. Komt de
schuldenaar dan nog niet over de brug, dan wordt de deurwaarder ingeschakeld.
Die \emph{re\"ele executie} vervangt dan vrijwillige nakoming \art{3:297}. Is voor
levering een notari\"ele akte vereist, dan kan de rechter zijn vonnis dezelfde
kracht geven \art{3:300}

\subsection{Schadevergoeding}

Toerekenbare tekortkoming in de nakoming kan leiden tot schadevergoeding
\art{6:74}:

Eisen:
\begin{enumerate}
  \item Tekortkoming in de nakoming
    \begin{enumerate}
      \item Verzuim \art{6:81} benodigd als nakoming nog mogelijk is
      \item Tekortkoming wanneer niet, te laat of ondeugdelijk nakomen.
    \end{enumerate}
  \item Toerekenbaarheid van de tekortkoming (bewijslast bij schuldenaar)
    \begin{enumerate}
      \item Verwijtbare schuld
      \item Risico op basis van:
        \begin{enumerate}
          \item Wet (hulppersonen, etc)
          \item Rechtshandeling (deel overeenkomst)
          \item Verkeersopvatting
        \end{enumerate}
    \end{enumerate}
  \item Schade
  \item Causaal verband: \emph{conditio sine qua non}
\end{enumerate}

\subsection{Toerekenbaarheid van de tekortkoming}

Een tekortkoming is toerekenbaar als gevolg van \emph{schuld} of
\emph{risico}. Zie hierboven.

Wanneer een schuldenaar door niet-toerekenbare tekortkoming verreikt wordt,
dan kan de schuldeiser het voordeel als schade vorderen \art{6:78}.

\subsection{Verzuim}

Verzuim treedt in door een ingebrekestelling \art{6:82}, tenzij de termijn
voor nakoming is verstreken of wanneer de schuldeiser mededeelt dat deze in
nakoming tekort zal schieten \art{6:83}.

Eisen verzuim:
\begin{enumerate}
  \item Nakoming is \emph{niet} blijvend onmogelijk, dan kan immers zonder
    verzuim \art{6:74} toegepast worden.
  \item Tekortkoming van de debiteur:
    \begin{enumerate}
      \item Opeisbare verbintenis
      \item Prestatie blijft uit of is ondeugdelijk
      \item Uitblijven prestatie is niet gerechtvaardigd door een
        opschortingsbevoegdheid
      \item Er is aan de eisen van \art{6:82-83} voldaan
    \end{enumerate}
  \item Toerekenbaarhied van de vertraging aan de debiteur.
\end{enumerate}

Voor het intreden van verzuim is in de regel een \emph{ingebrekestelling}
vereist. Hierin stelt de schuldeiser de schuldenaar een redelijke termijn voor
de nakoming \art{6:82 lid 1}. Als die termijn is verstreken dan raakt de
schuldenaar \emph{in verzuim}, met als gevolg onder meer een
schadevergoedingsplicht en een risico-omslag \emph{6:84}.

Er is geen termijn nodig wanneer de debiteur tijdelijk niet kan nakomen of uit
zijn houding blijkt dat aanmaning nutteloos zou zijn. \art{6:82 lid 2}

Het verzuim is afgelopen als de schuldenaar alsnog nakomt, met
schadevergoeding \art{6:86}. De schuldeiser kan ook de verbintenis tot
nakoming omzetten in een verbintenis tot vervangende schadevergoeding.
\art{6:86}.

\subsection{Ontbinding}

Wanneer de ene partij een verbintenis uit een wederkerige overeenkomst niet
nakomt, is de andere partij bevoegd de overeenkomst te ontbinden. \art{6:265}

De ontbinding kan zowel door een schriftelijke verklaring als door de rechter
worden uitgesproken \art{6:267}. Is de schuldenaar niet verhinderd na te
komen, dan is vereist dat de schuldenaar in verzuim is. \art{6:265 lid 2}

Ontbinding heeft \emph{geen terugwerkende kracht} \art{6:269}. Er ontstaat een
verbintenis tot ongedaanmaking van wat reeds is nagekomen.

\section{Onrechtmatige daad}

\begin{quote}

  Hij die jegens een ander een onrechtmatige daad pleegt, welke hem kan worden
  toegerekend, is verplicht de schade die de ander dientengevolge lijdt, te
  vergoeden. \art{6:162 lid 1}

\end{quote}

Onrechtmatige daad:
\begin{enumerate}
  \item Onrechtmatigheid
    \begin{enumerate}
      \item Inbreuk op een recht
        \begin{itemize}
          \item Persoonlijkheidsrechten
          \item Absolute rechten
        \end{itemize}
      \item Strijd met een wettelijke plicht
      \item Strijd met hetgeen volgens ongeschreven wet in het
        maatschappelijke verkeer betaamt. Bij gevaarzetting:
        \textbf{HR Kelderluik}:
        \begin{enumerate} \label{list:kelderluik}
          \item Mate van waarschijnlijkheid waarmee de niet-inachtneming van
            de vereiste oplettendheid en voorzichtigheid kan worden verwacht.
          \item De hoegrootheid van de kans dat daaruit ongevallen ontstaan
          \item De ernst van de gevolgen van een ongeval
          \item De mate van bezwaarlijkheid van te nemen
            veiligheidsmaatregelen.
        \end{enumerate}
    \end{enumerate}
  \item Toerekenbaarheid van de onrechtmatige daad aan de dader.
    \begin{itemize}
      \item Schuld
      \item Toerekenbaarheid krachtens wet of verkeersopvattingen
    \end{itemize}
  \item Schade
  \item Causaal verband tussen schade en daad
  \item Verband tussen het doel van de overtreden norm en het geschade belang:
    het relativiteitsbeginsel
\end{enumerate}

Een toerekenbare onrechtmatige daad wordt ook wel een \emph{fout} genoemd. Een
\emph{rechtvaardigingsgrond} neemt de onrechtmatigheid weg.

Rechtvaardigingsgronden:
\begin{itemize}
  \item Overmacht
  \item Noodweer
  \item Wettelijk voorschrift
  \item Ambtelijk bevel
  \item Toestemming van de benadeelde
\end{itemize}

In bepaalde gevallen is een onrechtmatige daad toerekenbaar ondanks het
ontbreken van (toerekenbaarheid van) schuld. Schuld kan aan ouders van
kinderen jonger dan 14 jaar worden toegerekend \art{6:169 lid 1}. Ook kan het
zijn dat ondanks het ontbreken van verwijtbaarheid, door een lichamelijke of
geestelijke stoornis, schade toegerekend kan worden \art{6:165}. Onervarenheid
en dwaling omtrent het objectieve recht of eigen bevoegdheid zijn gevallen van
toerekening op grond van \emph{in het verkeer geldende opvattingen}.

\subsection{Risicoaansprakelijkheid}

\subsubsection{Groepsaansprakelijkheid}

Wanneer men zich in een groep bevindt die onrechtmatig schade toebrengen, dan
is ieder van die groep hoofdelijk aansprakelijk, wanneer de kans op schade hen
van deelname aan die groep had moeten weerhouden \art{6:166 lid 1}.

Onderling moeten zij in gelijke delen aan de schadevergoeding bijdragen,
behoudens billijkheid in de omstandigheden van het geval \art{6:166 lid 2}.


\subsubsection{Aansprakelijkheid voor kinderen \art{6:169}}

Ouders zijn verantwoordelijk voor ``een als doen te beschouwen'' gedragingen
van hun kinderen onder de 14 jaar. Als het kind tussen de 14 en de 16 jaar oud
is, dan is de ouder \emph{naast} het kind aansprakelijk vanwege onvoldoende
toezicht, tenzij hij aantoont dat dat het hem niet kan worden verweten dat hij
de gedraging van het kind niet heeft belet.

\subsubsection{Aansprakelijkheid voor ondergeschikten \art{6:170}}

De werkgever is aansprakelijk voor \emph{fouten} van ondergeschikten, indien
door de opdracht de kans op de fout is vergroot en dat de werkgever
zeggenschap had over de gedragingen waarin de fout is gelegen. Dit is ook het
geval wanneer de onrechtmatige daad gepleegd is met zaken ter beschikking
gesteld door de werkgever of met behulp van gegevens verschaft door de
dienstbetrekking. \art{6:170 lid 1}

Voor personeel in dienst van een natuurlijk persoon is de aansprakelijkheid
meer ingeperkt. \art{6:170 lid 2}

\subsubsection{Aansprakelijkheid voor niet-ondergeschikten}

Iemand kan aansprakelijk zijn voor fouten van niet-ondergeschikten die in
opdracht van een bedrijf handelen, wanneer de fout werd begaan bij
werkzaamheden die bij die opdracht horen. \art{6:171}

Vertegenwoordigers kunnen ook aansprakelijk zijn voor fouten bij uitoefeningen
van een bevoegdheid tot vertegenwoordiging \art{6:172}.

\subsubsection{Aansprakelijkheid voor zaken, opstal en giftige stoffen}

Gebrekkige zaken die een bijzonder gevaar opleveren zijn voor risico van de
bezitter: \art{6:173}

De bezitter van een ondeugdelijk opstal is op dezelfde wijze aansprakelijk
voor schade: \art{6:174}. Er is een ``tenzij'': wanneer de bezitter bewijst
dat ook wetenschap van het gevaar geen onrechtmatige daad had opgeleverd.

Iemand die gevaarlijke stoffen gebruikt, is aansprakelijk voor schade die
erdoor onstaat \art{6:175}.

Een producent is aansprakelijk wanneer een product gebrekkig is \art{6:185 jo
186}.

\subsubsection{Aansprakelijkhid voor dieren}

De bezitter van een dier is aansprakelijk voor de door het dier aangerichte
schade, tenzij aansprakelijkheid had ontbroken wanneer hij het dier in de
macht had gehad. \art{6:179}


\section{Verbintenissen uit andere bron dan onrechtmatige daad of overeenkomst}

\subsection{Zaakwaarneming}

\emph{Zaakwaarneming} is het willens en wetens en op redelijke grond inlaten
met eens anders belang, zonder de bevoegdheid daartoe aan een rechtshandeling
of een elders in de wet geregelde rechtsverhouding te ontlenen. \art{6:198}.

Een zaakwaarnemer is bevoegd om in naam van de belanghebbende
rechtshandelingen te verrichten \art{3:201}. Zie ook
\ref{sec:vertegenwoordiging}.

Eisen zaakwaarneming:
\begin{enumerate}
  \item Behartiging van eens anders belang
  \item Willens en wetens
  \item Redelijke grond
  \item Noch krachtens rechtshandeling, noch krachtens elders in de wet
    geregelde rechtsverhouding bestaat een bevoegdheid tot de
    belangenbehartiging.
\end{enumerate}

\subsection{Onverschuldigde betaling}

Op de ontvanger van een onverschuldigde betaling rust de \emph{verbintenis}
tot ongedaanmaking van de betaling \art{6:203}.

\subsection{Ongerechtvaardigde verrijking}

Uit ongerechtvaardigde verrijking ontstaat een verbintenis tot
schadevergoeding \art{6:212}.

\section{Schadevergoeding}

Voor schadevergoeding komt in aanmerking \emph{vermogensschade en ander
nadeel} \art{6:95}. Ander nadeel is bijvoorbeeld persoonlijke schade, zie
\art{6:106}.

\subsection{Welke schade}

Schade wordt bepaald door de werkelijke toestand te vergelijken met de
hypothetische toestand, waarin men zou verkeren als de schadeveroorzakende
gebeurtenis achterwege zou zijn gebleven. Vergoeding van het \emph{positief
belang}, is wanneer de schade van het niet correct nakomen wordt vergoed. Men
wordt dan door schadevergoeding in een toestand gebracht alsof alles wel goed
was gegaan. Vergoeding van het \emph{negatief belang}, is de schade die niet
zou zijn ontstaan als de gebeurtenis niet zou hebben plaatsgevonden.

Het vast te stellen schadebedrag wordt in de regel op een \emph{abstracte}
manier vastgesteld, dat wil zeggen dat de theoretische schade wordt vergoed,
ookal laat men de schade bijvoorbeeld niet repareren. Dit is echter een
minimum, men mag hier van afwijken om een hoger bedrag te vorderen wanneer de
concrete schade hoger is.

\subsection{Causaal verband}

\subsubsection{Sine qua non}

Het \emph{sine qua non-verband} gaat over de vraag of de schade ook had
ingetreden als een bepaalde gebeurtenis niet had plaatsgevonden.

Wanneer er een \emph{kans} is dat iemand ergens aansprakelijk voor is, kan de
rechter veroordelen om een deel van de schade te vergoeden. Dit is het
leerstuk van de \emph{proportionele aansprakelijkheid}.

Deze vragen gaan echter vooral over het \emph{vestigen} van de
aansprakelijkheid.

\subsubsection{Oorzaak en gevolg: toerekening}

Om te bepalen \emph{welke} schade daadwerkelijk vergoed zal moeten worden,
moet bekeken worden welke schadeposten toerekenbaar zijn. Het gaat hier om een
\emph{causaal verband} tussen de schade en de gebeurtenis: is die
redelijkerwijs toerekenbaar? Deze afweging vindt plaats op grond van de aard
van de aansprakelijkheid, de aard van de gedragingen en de geschonden norm, de
aard van de schade, de verwijderdheid van het verband en de vraag of de schade
redelijkerwijs was te verwachten.

\subsection{Schadebeperking}

Alternatieve causaliteit \art{6:99}: bvb DES-casus; voordeelstoerekening
\art{6:100}, eigen schuld \art{6:101}, medeschuld \art{6:102}.

Voordeelstoerekening wordt slechts toegepast als dat redelijk is: een dief zal
die waarschijnlijk niet kunnen toepassen wanneer het voordeel bijvoorbeeld
komt uit vrijgevigheid (bvb collecte).

De rechter kan de vordering ook \emph{matigen} \art{6:109}.

\section{Overdracht}

Overdracht van goederen is het brengen van een goed in het vermogen van een
ander.

Eisen overdraagbaarheid \art{3:83 jo 84}:
\begin{enumerate}
  \item Overdraagbaarheid
  \item Geldige titel
  \item Beschikkingsbevoegdheid
  \item Levering
\end{enumerate}

\emph{Overdraagbaar} zijn in het algemeen \emph{eigendom}, \emph{beperkte
rechten} en \emph{vorderingsrechten} \art{3:83 lid 1}, hoewel de
overdraagbaarheid van vorderingsrechten kan worden \emph{uitgesloten}
\art{3:83 lid 2}.

\emph{Geldige titel} is vereist voor levering. Er is in Nederland een
\emph{causaal stelsel}, wat betekent dat als er een titelgebrek is, er geen
overdracht plaatsvindt. In de titel moet voldoende bepaald worden om welk goed
het gaat.

\emph{Beschikkingsbevoegdheid} komt neer op de regel dat niemand meer recht
kan overdragen dan hij zelf heeft.

\subsection{Levering}

\subsubsection{Levering van onroerende zaken}

De levering van onroerende zaken geschiedt door een daartoe bestemde, tussen
partijen opgemaakte notari\"ele akte, gevolgd door de inschrijving daarvan in
de openbare registers \art{3:89}. De eigendom gaat pas over zodra de akte is
ingeschreven en de levering daarmee voltooid.

In de transportakte moet voor onroerende goederen het goed met kadastrale
gegevens geindividaliseerd worden.

\subsubsection{Levering van roerende zaken}

In de regel worden roerende zaken geleverd door \emph{bezitsverschaffing}
\art{3:90}. Als de vervreemder echter niet de macht heeft over de zaak, dan
zal moeten worden geleverd met een akte \art{3:95}.

\subsubsection{Levering van vorderingen op naam}

De levering van een vorderingsrecht wordt \emph{cessie} genoemd. De
vervreemder heet dan \emph{cedent}, de verkrijger \emph{cessionaris}. De
debiteur wordt \emph{debitor cessus} genoemd.

De overdracht van een vordering op naam moet aan de eisen van \art{3:84}
voldoen. De levering heeft twee verschildende vormen, bepaald in \art{3:94}.

\emph{Openbare cessie} vereist een schriftelijke akte en mededeling aan de
schuldeiser \art{3:94 lid 1}. De overdracht is voltooid wanneer de mededeling
de schuldenaar heeft bereikt.

\emph{Stille cessie} hoeft niet te worden medegedeeld aan de schuldenaar, maar
er worden dan zwaardere eisen gesteld \art{3:94 lid 3}. Er is voor stille
cessie een authentieke akte vereist of een geregistreerde onderhandse akte.
Bij een stille sessie geldt de bescherming van \art{3:88 lid 1} pas zodra de
cessie is medegedeeld.

\subsection{Bezit en houderschap}

\emph{Bezit} is het houden van een goed voor zichzelf \art{3:107 lid 1}. Bezit
is onmiddelijk wanneer iemand een goed bezit zonder er dat iemand anders het
goed voor hem houdt \art{3:107 lid 2}. Wanneer dat wel het geval is, is het
bezit \emph{middelijk} \art{3:107 lid 3}.

\emph{Houderschap} is het houden van een goed voor een ander. Houderschap is
op dezelfde wijze als bezit middelijk of onmiddelijk \art{3:107 lid 4}.

\subsubsection{Bezitsverschaffing}

De belangrijkste manier om iemand het bezit over een zaak te verschaffen is
door die ander in staat te stellen de macht uit te oefenen over de zaak
\art{3:114}. Dit is veelal het fysiek overhandigen van de zaak.

Er zijn drie gevallen van bezitsverschaffing, waarbij de feitelijke macht niet
overgaat, maar de overdracht plaatsvindt door een tweezijdige verklaring
\art{6:115}:
\begin{description}

  \item[Levering constitutum possessorium \art{6:115 sub a}] De bezitter en
    houder wisselen van plaats.

  \item[Levering brevi manu \art{6:115 sub b}] Houder wordt bezitter.

  \item[Levering longa manu] Houder voor de \'e\'en wordt houder voor de
    ander.

\end{description}

Bij een \emph{eigendomsvoorbehoud} wordt de levering gedaan door de koper de
macht te verschaffen, maar het eigendom gaat pas over wanneer er betaald is.

\subsection{Derdenbescherming}

\subsubsection{art. 3:86 BW}

\begin{quote}

  Ondanks onbevoegdheid van de vervreemder is een overdracht overeenkomstig
  artikel 90, 91 of 93 van een roerende zaak, niet-registergoed, of een recht
  aan toonder of order geldig, indien de overdracht anders dan om niet
  geschiedt en de verkrijger te goede trouw is. \art{3:86 lid 1}

\end{quote}

Eisen voor een beroep op \art{3:86 lid 1}:
\begin{enumerate}
  \item Onbevoegdheid vervreemder
  \item Overdracht overeenkomstig \art{3:90, 91, 93}
  \item Roerende zaak, niet-registergoed of recht aan toonder of order
  \item Anders dan om niet
  \item Verkrijger te goeder trouw \art{3:11}
\end{enumerate}

Een geslaagd beroep op \art{3:86 lid 1} leidt tot eigendomsverkrijging,
ondanks onbevoegdheid van de vervreemder.

De eigenaar van een gestolen zaak kan haar in sommige gevallen gedurende drie
jaar terugvorderen \art{3:86 lid 3} ondanks de bescherming van lid 1. Als er
geen bescherming is, kan dat 20 jaar.

\subsubsection{art. 3:88 BW}

\begin{quote}

  Ondanks onbevoegdheid van de vervreemder is een overdracht van een
  registergoed, van een recht op naam, of van een ander goed waarop artikel 86
  niet van toepassing is, geldig, indien de verkrijger te goeder trouw is en
  de onbevoegdheid voortvloeit uit een eerdere overdracht, die niet het gevolg
  was van onbevoegdheid van de toenmalige vervreemder. \art{3:88}

\end{quote}

Stille cessionarissen kunnen zich veel minder gemakkelijk op dit artikel
beroepen. \art{3:94 lid 3, derde volzin}

\section{Executie en faillissement}

In de regel is er voor het verhaal zoeken op het vermogen van een schuldenaar
door de schuldeiser, voor die laatste een \emph{executoriale titel} vereist.
Wanneer geen executoriale titel vereist is noemt men het \emph{parate
executie}. Hiervan is bijvoorbeeld sprake bij \emph{pand} \art{3:248 lid 1} of
\emph{hypotheek} \art{3:268 lid 1}.

Het gehele vermogen van de schuldenaar strekt tot waarborg van zijn schulden
\art{3:276}. Dit kan via twee wegen: \emph{executoriaal beslag}, waarbij
op bepaalde goederen beslag wordt gelegd en die vervolgens openbaar worden
geveild, of via \emph{faillissement}, waarbij op het gehele vermogen van de
schuldenaar beslag komt te liggen ten gunste van de gezamelijke schuldeisers.

Zowel beslag als faillissement leiden tot beschikkingsonbevoegdheid van de
schuldenaar.

\section{De rangorde bij verhaal}

Bij verhaal kunnen mensen voorrang hebben op andere schuldeisers. Dit worden
\emph{preferente} vorderingen genoemd. Voorrang vloeit voort uit pand,
hypotheek en voorrecht en andere in de wet aangegeven gronden \art{3:278 lid
1}. Pand en hypotheek worden contractueel gevestigd, voorrechten ontstaan
alleen uit de wet.

Voorrecht, pand en hypotheek zijn \emph{nevenrechten}, wat betekent dat zij
bij overgang van de vordering van rechtswege overgaan op de nieuwe schuldeiser
\art{6:142}.

Vuistregels rangorde bij verhaal:
\begin{enumerate}
  \item Pand en hypotheek gaan voor \art{3:279}
  \item Speciale voorrechten gaan boven algemene voorrechten \art{3:280}
  \item Speciale voorrechten op hetzelfde goed hebben onderling gelijke rang
    \art{3:281 lid 1}
  \item Algemene voorrechten worden uitgeoefend in de volgorde waarin de wet
    hen plaatst \art{3:281 lid 2}
\end{enumerate}

Een retentor \art{3:290} mag met voorrang zich op de achtergehouden zaak
verhalen, maar heeft wel een executoriale titel nodig.

Een retentierecht kan ook worden ingeroepen tegen de schuldeisers van de
schuldenaar \art{6:53}. Ook tegen derden aan wie geleverd is kan het
ingeroepen worden \art{3:291 lid 1}.

\section{Pand en hypotheek}

Een pand- of hypotheekhouder is \emph{seperatist} bij een faillissement, wat
inhoudt dat hij zich mag verhalen op de bezwaarde zaak zonder zich iets van
het faillissement aan te trekken. Ook heeft de pand- of hypotheekhouder het
recht paraat te executeren.

Pand en hypotheek zijn afhankelijke rechten \art{3:7} en het zijn nevenrechten
\art{6:142}. Ze delen daardoor het lot van het recht waaraan zij zijn
verbonden.

Op alle goederen die overdraagbaar zijn kan een pand- of hypotheekrecht
gevestigd worden \art{3:228}. Als het recht is gevestigd op een registergoed,
is het een hypotheekrecht, anders is het een pandrecht \art{3:227 lid 1}.

\subsection{Vestiging van een pandrecht op een roerende zaak}

Via \art{3:98} wordt de vestiging van een beperkt recht gekoppeld aan de
regelingen met betrekking tot de overdracht van een goed. Ten aanzien van de
vestiging van een pandrecht is \art{3:84} van overeenkomstige toepassing.
Vereist zijn dus: geldige titel, beschikkingsbevoegdheid en overdraagbaarheid.
Levering is de vestigingsovereenkomst met de bijbehordende formaliteiten.

Het pandrecht wordt op een roerende zaak gevestigd door de zaak in de macht van de
pandhouder of een gezamelijk overeengekomen derde te brengen \art{3:236 lid
1}, \emph{vuistpand}. Het kan ook worden gevestigd door een authentieke of
geregistreerde onderhandse akte, zonder dat de zaak in de macht van de
pandhouder wordt gebracht, \emph{stil pand} \art{3:237 lid 1}.

Slechts de schuldeiser met een vuistpand wordt beschermd tegen
beschikkingsonbevoegdheid van de pandgever \art{3:238}.

\subsection{Vestiging van een pandrecht op een vordering op naam}

Een vordering op naam kan \emph{openbaar} of \emph{stil} verpand worden.

Openbare verpanding wordt gevestigd door een authentieke of onderhandse akte
\'en mededeling aan de schuldenaar \art{3:236 lid 2}.

Stille verpanding wordt gevestigd door een authentieke of geregistreerde akte
zonder mededeling aan de schuldenaar \art{3:239 lid 1}.

Voor enkele afwijkende regelingen, zie 13.5 van Nieuwenhuis.

Voor de regeling van hypotheek, zie ook Nieuwenhuis (lazy).

\section{Arresten}

\begin{description}

  \item[Haviltex] Bij de uitleg van de bepalingen in een contract komt het ook
    neer op ``de zin die pp.~in de gegeven omstandigheden over en weer
    redelijkerwijs aan deze bepalingen mochten toekenen en op hetgeen zij te
    dien aanzien redelijkerwijs van elkaar mochten verwachten. Daarbij kan
    mede van belang zijn tot welke maatschappelijke kringen pp.~behoren en
    welke rechtskennis van zodanige pp.~kan worden verwacht.''

  \item[Hofland/Hennis] Uitnodiging tot het doen van een aanbod

  \item[Kelderluik] Zie Eisen onrechtmatige daad, \ref{list:kelderluik}.

  \item[Kribbebijter] Vertegenwoordigingskwaliteit hangt af van de wederzijdse
    verklaringen en gedragingen en wat daaraan afgeleid is / mocht worden
    afgeleid.

  \item[Pos/Van den Bosch] Schade aan andermans belang toebrengen kan
    onrechtmatig zijn.

  \item[Tandarts] Bij onrechtmatige daad ``in strijd met de wet'', moet die
    wet wel tot doel hebben de belangen te beschermen van de eiser.

  \item[Val Dias/Salters] Verklaring of gedraging van pseudo-volmachtgever in
    \art{3:61 lid 2} kan ook een nalating zijn.

\end{description}



\end{document}

% vim: set ts=4 sw=2 tw=78 formatoptions=t1 et :
