\documentclass{article}

\usepackage{amsmath}
\usepackage{amsthm}
\usepackage{amssymb}
\usepackage{subcaption}
\usepackage{cancel}
\usepackage{amsfonts}
\usepackage{a4wide}
\usepackage{microtype}
\usepackage{hyperref}

\newtheorem{definition}{Definition}
\newtheorem{lemma}{Lemma}

\newcommand{\wmod}[1]{\pmod{#1}}
\newcommand{\ZZ}{\mathbb{Z}}
\newcommand{\Z}{\ZZ}
\newcommand{\NN}{\mathbb{N}}
\newcommand{\N}{\NN}
\newcommand{\QQ}{\mathbb{Q}}
\newcommand{\CC}{\mathbb{C}}
\newcommand{\RR}{\mathbb{R}}
\newcommand{\FF}{\mathbb{F}}
\newcommand{\GF}{\operatorname{GF}}
\newcommand{\egcd}{\operatorname{xgcd}}
\newcommand{\xgcd}{\operatorname{xgcd}}
\newcommand{\mchar}{\operatorname{char}}

\newcommand{\ZZx}[1]{(\ZZ_{/#1})^\times}

\author{Thom Wiggers}
\title{Lecture notes Cryptology 2015}

\begin{document}
\maketitle

\section{Lecture 1 and 2}

Cryptology consists of cryptography (constructive) and cryptanalysis (destructive).

\subsection{Modular arithmetic}

$\ZZ$ is the integers $\{\dots, -1, 0, 1, \dots\}$.

$\ZZ/n$ is the residue class of $\ZZ$ modulo $n$.

$$\ZZ/6 = \{0, 1, 2, 3, 4, 5, 6\} = \{-2, -1, 0, 1, 2, 3\}$$

We use integers to represent classes, so $0$ stands for the set of integers
congruent to $0 \mod 6$, i.e., $\{0, 6, 12, 18, \dots, -6, -12, \dots\}$.

$7 \equiv 1 \pmod{6}$ ($\hat{=} 7$ is congruent to $1 \pmod{6})$.

We usually want the samllest representative on the right, depends on the set chosen,
e.g. $17 \equiv 5 \equiv -1 \mod 6$.

\subsection{Multiply mod $n$}

When computing $a \cdot b \mod n$ we can compute and then reduce or reduce at any intermediate
step.

\[ 17 \cdot 35 \equiv -1 \cdot -1 \equiv 1 \pmod{6} \]

This is especially important for exponentiation $a^b \pmod{n}$.

\subsubsection{Square and multiply}

\begin{verbatim}
c = a
l = length of b as binary string
for i in reversed(range(l-2)):
  c = c^2 mod n
  if b_i == 1:
    c = c * a mod n
return c
\end{verbatim}

\subsection{Inverse mod $n$}

Divide equation mod 7 by 4, this means undoing a multiplication by 4, i.e. find
an integer mod 7 such that $a \cdot 4 \equiv 1 \pmod 7 = a^{-1} \equiv 4 \pmod
7$. Here $a \equiv 2 \pmod{7}$ ($2 \cdot 4 \equiv 8 \equiv 1 \pmod{7})$.

In general, use the Extended Euclidian Algorithm ($\xgcd$) to compute $a$.

\[
  \egcd(n,m) = \begin{cases}
    \egcd(m,n) & m > n \\
    \langle 1,0 \rangle & n \mod m = 0 \\
    \langle y, x - (y \cdot (n \operatorname{div} m))\rangle & \langle x, y \rangle = \egcd(m, n \mod m) \\
  \end{cases}
\]

See \ref{sec:rsaexample} for an example how to write this.

Inverses do not exist if the numbers are not coprime, e.g. $4$ is not invertible
mod 6 (common factor 2).

The set of invertible integers mod $n$ is denoted $\ZZx{n}$. Example: 
$\ZZx{7} = \{1, 2, 3, 4, 5, 6\}$, with inverses: $1^{-1} \equiv 1$, 
$2^{-1} \equiv 4$, $3^{-1} \equiv 5$, $6^{-1} \equiv 6$. These
are exactly the integers coprime 7.

$\ZZx{6} = \{1, 5\}$.

$\ZZx{p} = \{1, 2, \dots, p-1\}$ for $p$ prime.

Eulers phi gives the size of $|\ZZx{n}| = \phi(n)$.

$\phi(7) = 6, \phi(6) = 2$.

\[ \phi(p) = p-1 \text{ for $p$ prime} \]

\[ \phi(pq) = (p-1)(q-1) \text{ for $p, q$ prime}\]

\[ \phi(p^b) = p^b - p^{b-1} \]

\[ n = \prod\limits_{i=0}^{l-1} p_i^{e_i}, p_i \ne p_j \text{ then } \phi(n) = \prod (p_i^{e_i} - p_i^{e_i -1}) \]

\subsection{Lagrange's Theorem}

Let $G$ be a finite group of size $|G| = l$ then for any element $a \in G$ we have $a^l = 1$, where
$1$ is the neutral element and $G$ is written multiplicatively.

eg. $a^6 \equiv 1 \pmod{7}$ for $a \in \ZZx{7}$.

\subsection{RSA}

Pick two large primes $p, q$, with $p \ne q$. Put $n = pq$ and compute $\phi(n) = (p-1)(q-1)$.
Pick an $e$ coprime $\phi(n)$. Compute $d \equiv e^{-1} \pmod{\phi(n)}$ using $\xgcd$.

\subsection{RSA parameters example}\label{sec:rsaexample}

$p = 5, q = 7, n = 5 \cdot 7 = 35, \phi(35) = (5 - 1)(7-1) = 24$. Pick $e = 5$, $\gcd(5, 24) = 1$.
$5^{-1} \equiv 5 \pmod{24}$ so $d=5$.

Other option $e = 7$ with $\xgcd$:

$$\begin{array}{cccr}
    24 & 1 & 0 &                              \\
    7  & 0 & 1 & \text{quotient 24 div 7 = 3} \\
    \cline{1-3}
    3  & 1 & -3 & 7 \operatorname{div} 3 = 2 \\
    \cline{1-3}
    1 & -2 & 7 & \\
    r & a  & b &
\end{array}$$

Every row satisfies $r = 24 \cdot a + 7 \cdot b$.

Then $p_k = (5, 35), \, s_k = 5$ (or $(7, 35), d=7$).

\subsection{Encryption}
Message $m < n, m \in \NN$.

\[ c \equiv m^e \pmod{n} \]

\subsection{Decryption}

\[ m \equiv c^d \pmod{n} \]

This works because $m' \equiv c^d \equiv m^{e^d} \equiv m^{ed}$ where
$ed \equiv 1 \pmod{\phi(n)}$ i.e. $1 + k\phi(n)$ for some $k$.
\[ m' \equiv m^{1 + k \phi(n)} \equiv m \cdot 1 \pmod{n} \] via Lagrange:
in $\ZZx{n}$, $m^{\phi(n)} \equiv 1 \pmod{n}$.

Check that $m^{1 + k\phi(n)} \equiv m \pmod{n}$ even when $\gcd(m, n) \ne 1$.

To show this we use the \emph{Chinese Remainder Theorem}.

$m \equiv 0 \pmod{p}, m \equiv a \pmod{q}$, $a \ne 0$.

Then $m^{ed} \equiv 0^{ed} \equiv 0 \pmod{p}$. In $\ZZx{q}$ we 
have $a^{q-1} \equiv 1 \pmod{q}$, thus $m^{ed} \equiv a^{1 + k\phi(n)} \equiv a^{1 + k(q-1)(p-1)}
\equiv a \cdot (a^{q-1})^{p-1} \equiv a \cdot (1)^{p-1} \equiv a \pmod{q}$.

So:
\begin{align*}
    m^{ed} &\equiv 0 \equiv m \pmod p \\
    m^{ed} &\equiv a \equiv m \pmod q 
\end{align*}
thus the CRT gives $m^{ed} \equiv m \pmod{n}$.

For the case $0^{ed} \equiv 0 \pmod{n}$.

So RSA gives $m' = m$.

\subsection{RSA-CRT}

For fast encryption, we choose a small $e$ when generating the key.

Choosing small $d$ gives Eve your key, but we can decrypt faster if we
use RSA-CRT.

Compute \begin{align*}
    c^d \equiv m_p \pmod p \\
    c^d \equiv m_q \pmod q
\end{align*}

This is faster because the operands are smaller (naively\footnote{``Naively''
because Karatsuba double-length multiplication costs only 3 times as much and
for very long integers the FFT takes only twice as long} this saves a factor of
$4$ in each computation, do this twice, so save a factor of 2. Combine $m_p$ and
$m_q$ using CRT to get m, this needs just one multiplication given $u = p^{-1}
\pmod q$, $v = q^{-1} \pmod p$. Then $m = m_p \cdot q \cdot v + p \cdot u \cdot
m_q$.

\section{Lecture 3 Finite Fields}

\begin{definition}
    A set $K$ is a \textbf{field} with respect to $\circ$ and $\diamond$, denoted $(K, \circ, \diamond)$ if:

    \begin{enumerate}
        \item $(K, \circ)$ is een abelian group.
        \item $(K^* = K\setminus \{e_\circ\}, \diamond)$ is an abelian group.
        \item Distributivity holds in $K$: $a \diamond (b \circ c) = a \diamond b \circ a \diamond c$.
    \end{enumerate}

\end{definition}

\begin{definition}
    A group $(K, \Box)$ is \textbf{abelian} if these properties hold:

    \begin{description}
        \item[Closure] $\forall a, b \in K, \, a \Box b \in K$.
        \item[Associativity] $\forall a,b \in K, \, (a \Box b) \Box c = a \Box (b \Box c)$
        \item[Identity] $\exists e_\circ \in K, \, \forall a \in K, \,  a \Box e_0 = a$
        \item[Invertibility] $\forall a \in K, \, \exists b \in K, \, a \Box b = e_\circ$
        \item[Commutativity] $\forall a, b \in K, \, a \Box b = b \Box a$.
    \end{description}
\end{definition}

Examples of fields and non-fields:
\begin{tabular}{ll}
    $(\NN, +, \cdot)$ & isn't a field: no negative numbers \\
    $(\ZZ, +, \cdot)$ & isn't a field: no inverse          \\
    $(\QQ, +, \cdot)$ & is a field                         \\
\end{tabular}

\begin{definition}
    If $(K, \circ, \diamond)$ and $(L, \circ, \diamond)$ are fields and $K \subseteq L$
    then $K$ is a \textbf{subfield} of $L$.
\end{definition}

\begin{definition}
    Let $L$ be a field and $K$ be its subfield. The \textbf{extension degree}, denoted as $[L:K]$
    is defined as $\operatorname{dim}_K L$, the dimension of $L$ as a $K$ vectorspace.
\end{definition}

\begin{definition}\label{def:characteristic}
    Let $K$ be a field. The \textbf{characteristic} of $K$, $\operatorname{char}(K)$ is the
    smallest positive integer $m$ such that
    $\underbrace{e_\diamond \circ e_\diamond \circ \dots \circ e_\diamond}_{m} = e_\circ$.
    If there is no such $m$, then $\operatorname{char}(K) = 0$.
\end{definition}

\begin{lemma}
    The characteristic of a field is either 0 or prime.
\end{lemma}

\begin{proof}
    Let $(K, \circ, \diamond)$ be a field with $\mchar(K) = n$ and
    $n > 0$. Assume $n$ isn't prime so $n = a \cdot b$ with $1 < a, b <n$.

    Then we have
    \begin{equation}\label{eq:proofcharprime1}
        e_\circ = [n]e_\diamond = [a]e_\diamond \diamond [b]e_\diamond.
    \end{equation}

    We can show this by
    \begin{align*}
          & [a] e_\diamond \diamond [b]e_\diamond                                             \\
          & = (e_\diamond \circ [a-1] e_\diamond) \diamond [b] e_\diamond                     \\
          & = e_\diamond \diamond [b]e_\diamond \circ [a-1] e_\diamond \diamond [b]e_\diamond \\
          & = \cancel{e_\diamond} \diamond [b]e_\diamond \circ [a-1] e_\diamond \diamond [b]e_\diamond & \mbox{(identity)} \\
          &\vdots \\
          & = \underbrace{[b] e_\diamond \circ [b] e_\diamond \circ \dots [b] e_\diamond}_\text{$a$ times} \\
          & = [a \cdot b] e_\diamond
    \end{align*}

    So \eqref{eq:proofcharprime1} is true, so either $[a]e_\diamond = e_\circ$ or $[b]e_\diamond = e_\circ$.
    But by Definition \ref{def:characteristic} then $a$ or $b$ would be the characteristic of $K$. So the
    characteristic can't be the product of two integers.

    So any characteristic $n > 0$ must be prime. The characteristic can be 0 by
    Definition \ref{def:characteristic}. So the characteristic must be either
    0 or prime.
\end{proof}

\begin{lemma}
    A finite field $K$ has $\mchar(K) \ne 0$, so $\mchar(K)$ is prime.
\end{lemma}
\begin{proof}
    Since $K$ is finite, there must be $i, j \in \NN$ with $[i]e_\diamond = [j]e_\diamond$.
    Let $i > j$, then $[i-j] e_\diamond = e_\circ$. And so $\mchar(K)|(i-j)$.
\end{proof}

A field $K$ with $\mchar(K) = p$ forms a ring inside the field as
$e_\diamond, [2]e_\diamond, \dots, [p-1] e_\diamond, e_\circ, e_\diamond, \dots$.

A finite field maps to $\ZZ/p\ZZ$ as $[i] e_\diamond \mapsto i + p\ZZ$ or $[i]
e_\diamond \mapsto i \mod p$.

\begin{definition}
    A \textbf{primefield} is the smallest subfield contained in a field $K$.
\end{definition}

\begin{lemma}
    Let $K$ be a finite field of $\mchar(K) = p$. Then there is a primefield of
    $K$ that is isomorphic to $\ZZ/p\ZZ$.

    \begin{align*}
        e_\circ \mapsto 0 & \text{ in $\ZZ/p\ZZ$} \\
        e_\diamond \mapsto 1 & \text{ in $\ZZ/p\ZZ$} \\
    \end{align*}

    The vectorspace as linearly independant vectors $\alpha_1, \alpha_2, \dots,
    \alpha_n$ with n the dimension. Every element in the field can be written as
    \[ \sum^n_{i=1} c_i \alpha_i \text{ with } c_i \in \ZZ/p\ZZ.\]

\end{lemma}


\begin{lemma}
    Let $K$ be a finite field. There exists a prime $p$ and an integer $n \in \NN_{>0}$ such
    that $|K| = p^n$ and $\mchar(K) = p$. (\emph{Notation:} $\FF_{p^n}$ or $\GF(p^n)$.)

    \[ \left(\sum_{i=1}^n c_i \alpha_i\right) + \left(\sum_{i=1}^n d_i \alpha_i\right)
        = \sum_{i=1}^n (c_i + d_i)\alpha_i \]

    so $+ \to \circ$ and $\cdot \to \diamond$.

\end{lemma}

\begin{lemma}
    Let $K$ be a finite field. The multiplicative group $K^*$ is cyclic.

    Thus, for every $a \in K^*$ we have $a^{p^{n}-1} = 1$.
\end{lemma}
\begin{proof}
    \[ a^{p^n} = a \Leftrightarrow a \cdot a^{p^{n}-1} = 1 \cdot a \Leftrightarrow a^{p^{n}-1} = 1 \]
\end{proof}

\begin{definition}
    \textbf{Polynomial ring over field $K$}

    \begin{align*}
        K[x] &= \left\{ \sum_{i=1}^n a_i x^i | n \in \NN, a_i \in K \right\} \\
        f &\in K[x] \\
        f &= \sum f_i x^i
    \end{align*}

\end{definition}

\begin{definition}
    A polynomial $f \in K[x]$ is called \textbf{irreducible} if $\deg(f) \ge 1$ and it cannot be
    written as the product of polynomials with a smaller degree.
\end{definition}

Some examples
\begin{align*}
    x^2 - 1 & = (x-1)(x+1) & \text{reducible in $\RR[x]$}  \\
    x^2 + 1 &              & \text{irreducible in $\RR[x]$} \\
    x^2 + 1 & = (x - i)(x + i) & \text{reducible in $\CC[x]$}
\end{align*}

\section{Lecture 4: finite fields continued}

An example of $\GF(4)$ with symbols can be found in Table \ref{tab:gf4examplesymb}.

\begin{table}[h]
    \centering
    \begin{subfigure}[b]{.25\textwidth}
        \begin{tabular}{c|cccc}
            $+$ & $\Box$ & $\circ$ & $\star$ & $\bigtriangleup$ \\ \hline
            $\Box$ & $\Box$ & $\circ$ & $\star$ & $\bigtriangleup$ \\
            $\circ$ & $\circ$ & $\Box$ & $\bigtriangleup$ & $\star$ \\
            $\star$ & $\star$ & $\bigtriangleup$ & $\Box$ & $\circ$ \\
            $\bigtriangleup$ & $\bigtriangleup$ & $\star$ & $\circ$ & $\Box$
        \end{tabular}
    \end{subfigure}
    \begin{subfigure}[b]{.25\textwidth}
        \begin{tabular}{c|ccc}
            $\cdot$ & $\circ$ & $\star$ & $\bigtriangleup$ \\ \hline
            $\circ$ & $\circ$ & $\star$ & $\bigtriangleup$ \\
            $\star$ & $\star$ & $\bigtriangleup$ & $\circ$ \\
            $\bigtriangleup$ & $\bigtriangleup$ & $\circ$ & $\star$
        \end{tabular}
    \end{subfigure}
    \caption{An example of $\GF(4)$ with addition and multiplication}
    \label{tab:gf4examplesymb}
\end{table}

$\GF(4) = \GF(2^2) = \GF(2)^2$

\[\GF(4) = \left( \left\{
    \begin{pmatrix} 0 \\ 0 \end{pmatrix},
    \begin{pmatrix} 0 \\ 1 \end{pmatrix},
    \begin{pmatrix} 1 \\ 0 \end{pmatrix},
    \begin{pmatrix} 1 \\ 1 \end{pmatrix},
\right\}, +, \cdot \right). \]

Basis:

    \[ \alpha_1 = \begin{pmatrix} 0 \\ 1 \end{pmatrix} \qquad
       \alpha_2 = \begin{pmatrix} 1 \\ 0 \end{pmatrix} \]
\begin{align*}
    \begin{pmatrix} 0 \\ 0 \end{pmatrix} &= 0 \cdot \alpha_1 + 0 \cdot \alpha_2 = 0 &
    \begin{pmatrix} 1 \\ 0 \end{pmatrix} &= 0 \cdot \alpha_1 + 1 \cdot \alpha_2 = \alpha_2 \\
    \begin{pmatrix} 0 \\ 1 \end{pmatrix} &= 1 \cdot \alpha_1 + 0 \cdot \alpha_2 = \alpha_1 &
    \begin{pmatrix} 1 \\ 1 \end{pmatrix} &= 1 \cdot \alpha_1 + 1 \cdot \alpha_2 = \alpha_1 + \alpha_2
\end{align*}

We can instead use basis $1, a$.
\begin{align*}
    0 \cdot a + 0 \cdot 1 &= 0 &
    1 \cdot a + 0 \cdot 1 &= a \\
    0 \cdot a + 1 \cdot 1 &= 1 &
    1 \cdot a + 1 \cdot 1 &= a + 1
\end{align*}

This base vector representation is nice for addition but does not help for
multiplication.

\section {Security course}
\section{Security Goals}

\begin{description}
  \item[Confidentiality] Third parties cannot read what $A$ and $B$ are saying.
  \item[Integrity] Third parties cannot modify the contents of the communication.
  \item[Authenticity] $A$ and $B$ are certain about each other's identities.
  \item[Availability] Third parties cannot prevent $A$ and $B$ from communicating.
  \item[Non-repudiation] Someone cannot deny what they have communicated.
  \item[Accountability] Reliable log of communication history.
\end{description}

\section{Crypto primitives}

\begin{tabular}[h]{c||p{100pt}|l|l}
  \bf \# Keys & \bf Name & \bf Key names & \bf Notation \\ \hline \hline
  0 & Hash functions & --- & $h(m)$ \\ \hline
  1 & Symmetric Crypto & Shared, secret & $K\left\{ m \right\}$  \\\hline
  2 & Asymmetric Crypto (public key crypto) & Public \& private keypair & $\left\{ m \right\}_K$ \\
\end{tabular}

\section{Active Attacks}

\subsection{Replay attack}

Intercepted data is sent again. Countermeasures: nonces, timestamps.

\subsection{Reflection attack}

Data from one session is reused in another session, to for example make the
target do encryption/decryption work.

\subsection{Man-in-the-middle attacks}

Also passive: relay attack. Active involves reencryption. countermeasure:
strong authentication.

\section{Cipher modes}

\subsection{Electronic Code Book (ECB)}
$m = m_0m_1m_2\cdots \rightarrow C = K\{m_0\},K\{m_1\},K\{m_2\}\cdots$.

Attack vectors: occurrence frequency, swappping can go unnoticed easily.

\subsection{Cipher Block Chaining (CBC)}

\begin{align*}
  c_0     &= K\{m_0 \oplus IV\} \\
  c_{n+1} &= K\{m_{n+1} \oplus c_n\} \\
\end{align*}

$IV$ may be sent openly or be constant.

One garbled block means two blocks are lost in deciphering. Last block
can be used to verify integrity.

\subsection{Output Feedback Mode (OFB)}

First, pick a random number (Initialisation Vector, $IV$) and use it to
create a keystream:

\[ K\{IV\},K\{K\{IV\}\}, K\{K\{K\{IV\}\}\} \cdots \]

Then XOR with incoming bitstream. Garbled bits are lost, but only those bits.
If sender/receiver are out of sync, everything is lost.

Variation: $c_n = m_n \oplus K\{IV + n\}$ (Counter mode).

\subsection{Cipher Feedback (CFB)}

\begin{align*}
  c_0 &= IV \\
  c_{n+1} &= K\{c_n\} \oplus m_{n+1}
\end{align*}

\section{Symmetric Crypto}

\subsection{Basic Techniques}

\begin{enumerate}

  \item \textbf{Substitution}: Swapping characters from the alphabet. Key $K$
    is the subsitution function.

  \item \textbf{Transposition}: Changing positions of characters (by block).
    $K$ is the position exchange function.

  \item \textbf{One-time pad}: Take bitwise XOR of message with keystream.
    $K$ is keystream of at least the same length as the message.

\end{enumerate}

One-time pads are sometimes generated using linear feedback shift registers. See
slide 25.

A downside of symmetric crypto is that one needs ${N \choose 2}
= \frac{N(N-1)}{2}$ keys if $N$ people want to communicate pairwise securely.

Also, if key $K$ is lost by $A$, $B$ is also affected.

\subsection{Basic Protocols}

\subsubsection{Integrity}

\[
  A \longrightarrow B : m, K_{AB}\{h(m)\}
\]

Hash function for efficiency.

$B$ can verify integity by decrypting and comparing $h(m)$ with the hash he
creates from $m$.

\subsubsection{Confidentiality}

\[ 
  A \longrightarrow B : K_{AB}\left\{ m \right\}
\]

Only those with $K_{AB}$ can read this, obviously.

When combined with integrity ($A \longrightarrow B : K\{m, K\{h(m)\}\}$) it is
important to use different keys, because otherwise if $K$ is compromised, both integrity
and confidentiality are broken.

\subsubsection{Authenticity}

``Shared Secret'': You can be authenticated by something only you and the other
party know. Problem: secret used in the clear.

It is better to send riddles that can only be solved efficiently using the
secret key. The riddle needs to be fresh every time (against replay attacks).
Often achieved by using \emph{nonces}.

\begin{align*}
  A & \longrightarrow  B  : A, N_A             \\
  B & \longrightarrow  A : K_{AB}\{N_A, N_B\} \\
  A & \longrightarrow  B  : N_B                \\
\end{align*}

\section{Hashing}

\subsection{Properties}

\begin{description}

  \item[Preimage Resistant (one-way)] Given hash value $x$, it should be 
    hard to find $m$ with $h(m) = x$.

  \item[Second preimage resistant] Given an $m$, it should be hard to find 
    $m' \neq m$ with $h(m) = h(m')$.

  \item[Collision resistant] It should be hard to find \emph{any} pair $m \neq
    m'$ with $h(m) = h(m')$.

\end{description}


\subsection{Non-revealing commitment}

e.g. for flipping coins one can use:

\begin{align*}
  A & \longrightarrow B : h(C_A, N_A) \\
  B & \longrightarrow A : h(C_B,N_B)  \\
  A & \longrightarrow B : C_A, N_A    \\
  B & \longrightarrow A : C_B, N_B    \\
\end{align*}

(nonces are used to prevent cheating by lookup table - coin outcomes are very
limited)

\subsubsection{Lamport's hash}

$C$ has for each user $A$ a pair $[n \in \mathbb{N}, h^n(\text{passwd}_A)]$.

\begin{align*}
  A & \rightarrow C : A \\
  C & \rightarrow A : n \\
  A &\rightarrow C  : h^{n-1}(\text{passwd}_A) = x
\end{align*}

$C$ can then verify the authenticity of $A$ by checking $h(x)
= h^n(\text{passwd}_A)$. $C$ can also then set a new pair $[n-1,x]$ or
$[n+1, h(h(x))]$.

\section{Asymmetric Crypto}

Using two keys: one for decryption (private key, $K_d$) and one for encryption
(public key, $K_e$).

\begin{itemize}
  \item Encryption: $\{m\}_{K_e} $
  \item Decryption: $[c]_{K_d}$.
\end{itemize}

Identity function: 
\[
  [\{m\}_{K_e}]_{K_d} = m
\]

Though for some systems $\{[m]_{K_d}\}_{K_e} =m$ is also valid, this is not
a requirement.

\section{Number theory}

\subsection{Equivalence classes}

\[ \forall N \in \N, n \in \Z, m \in \Z 
  \,[n \equiv m \wmod{N} \Leftrightarrow 
      \text{ there is a } k \in \Z \text { with } n = m + k \cdot N]
\]
$\Z_N$ is the set of numbers modulo $N$. Thus 
\[ \Z_N = \{ 0, 1, \dots N-1\} \]
For every $m \in \Z$ we have $m \mod N \in \Z_N$.

If a product modulo $N$ is 1, for instance, $4 \cdot 4 \equiv 1 \wmod{15}$, you
can say $\frac{1}{4}=4 \wmod{15}$. For $\Z_p$ where $p$ is prime, every
non-zero number $n \in \Z_p$ has an inverse $\frac{1}{n} \in \Z_p$.

A finite cyclic group often has a generator $g$ that $g^n = k$ so that $k$ can become
any number in the group.

\subsection{Greatest Common Divisor}

$\gcd(n,m) = $ greatest $k$ which divides both $n$ and $m$. If $\gcd(n,m) = 1$, one
calls $n,m$ relative prime.

\[ 
  \gcd (n,m) = \begin{cases}
    \gcd(m,n) & m > n \\
    n & m=0 \\
    \gcd(m, n \operatorname{mod} m) & \text{otherwise}\\
  \end{cases}
\]

\subsection{Extended GCD}

\[
  \egcd(n,m) = \begin{cases}
    \egcd(m,n) & m > n \\
    \langle 1,0 \rangle & n \mod m = 0 \\
    \langle y, x - (y \cdot (n \operatorname{div} m))\rangle & \langle x, y \rangle = \egcd(m, n \mod m) \\
  \end{cases}
\]

Results in the solutions $a,b$ for $a \cdot m + b \cdot n = \gcd(m,n)$.

\section{RSA}

RSA has a public key $e$, a private key $d$ and a modulo $n=p\cdot q$. \\
Public key: $(n, e)$\\
Private key: $(n,d)$ \\

\subsection{RSA key creation}
\begin{enumerate}
  \item Pick two primes $p$ and $q$. $n=pq$
  \item Calculate $\phi(n) = (p-1)(q-1)$.
  \item Pick $e$ from $\Z^*_{\phi(n)}$.
  \item Calculate $d$ from $\frac{1}{e} \in \Z^*_{\phi(n)}$ using $\egcd(\phi(n),e)$.
\end{enumerate}

\subsection{Encryption / Decryption}
Encrypt using RSA by $c = m^e \mod n$\\
Decrypt using $m = c^d \mod n$.

\subsection{Signatures with RSA}

\[
  A \longrightarrow B : m, [h(m)]_{d_A}
\]

Works because RSA is symmetric and since only $A$ has access to $d$, only he
can have encrypted it.

\subsubsection{Blind signatures}

\begin{enumerate}
  \item $A$ computes $m' = r^e \cdot m \mod n$, where $r$ is a random number, and gives
    this to $B$.
  \item $B$ signs $m'$, giving $k = [m']^d \mod n$ to $A$.
  \item $A$ computes $\frac{k}{r} = \frac{r^{ed} \cdot m^d}{r} = \frac{r \cdot m^d}{r} = m^d \mod n$
\end{enumerate}

Can be used for anonymous signatures on e-cash, e-voting\dots

\subsection{Certificates}
A certificate is a public key that has been signed by a thrusted third party.

\section{Diffie-Hellman Key Exchange}

Using a generator $g$ in a finite cyclic group:

\begin{align*}
  A &\longrightarrow B: A, g^{s_A} \\
  B &\longrightarrow A: B, g^{s_B} \\
\end{align*}

Then $K_{AB} = g^{s_A s_B} = (g^{s_A})^{s_B} = (g^{s_A})^{s_B}$.

This is not authentication!

\section{El-Gamal}
Fix a generator $g \in G$ of size $N$. \\
Private key: $n \in \mathbb{N}$ with $n < N$. \\
Public key: $h = g^n \in G$.

\subsection{Encryption / Decryption}
\subsubsection{Encryption}

\begin{enumerate}
  \item Represent message as $m \in G$.
  \item Choose a random $r < N$.
  \item $\{m\}_h = (g^r, m \cdot h^r)$.
\end{enumerate}

\subsubsection{Decryption}
\begin{enumerate}
  \item Ciphertext $c = (c_1,c_2)$ with $c_i \in G$.
  \item \[
      [(c_1, c_2)]_n = \frac{c_2}{(c_1)^n} = \frac{m \cdot h^r}{(g^r)^n} = \frac{m \cdot (g^n)^r}{(g^r)^n} = m
    \]
\end{enumerate}


\subsubsection{Signatures}

Choose random $r < p-1$ in $\Z^*_p$ with $\gcd(r,p-1) = 1$ and then:
\[
  \operatorname{sign}_n(m) = \left( g^r, \frac{H(m) - n \cdot g^r}{r} \mod p-1 \right)
\]

Verification of signature $(s_1, s_2)$:
\[
  g^{H(m)} \overset{??}{=} s_1^{s_2} \cdot h^{s_1}
\]


\end{document}
